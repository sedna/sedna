% AdminGuide.tex: Sedna Administration Guide
% Copyright (C) 2010 ISP RAS
% The Institute for System Programming of the Russian Academy of Sciences

\documentclass[a4paper,12pt]{article}

\usepackage{alltt}         % Like verbatim but supports commands inside
\usepackage{theorem}
\newtheorem{note}{Note}    % To insert notes
\usepackage{multirow}      % Allows inserting tables
\usepackage{ifpdf}         % Package for conditionals in TeX
\newcommand{\TocAt}[6]{}   % To avoid processing \TocAt by LaTeX

\title{Sedna Administration Guide}
\date{}

% Switch for between PDF and other formats to generate bookmarks,
% pretty table of contents and set document's information in PDF
\ifpdf
  \usepackage[colorlinks=true, linkcolor=blue,
              citecolor=blue, urlcolor=blue,
              pdftex,                %%% hyper-references for pdflatex
              bookmarks=true,        %%% generate bookmarks ...
              bookmarksnumbered=true %%% ... with numbers
  ]{hyperref}
  \pdfadjustspacing=1
  \hypersetup{
	pdfauthor   = {Sedna Team},
	pdftitle    = {Sedna Administration Guide}
  }
\else
  \usepackage[colorlinks=true, linkcolor=blue,
			  citecolor=blue, urlcolor=blue]{hyperref}
\fi

% Use citemize environment to produce tightly packed lists
\newenvironment{citemize}
{\begin{itemize}
  \setlength{\itemsep}{0pt}
  \setlength{\parskip}{0pt}
  \setlength{\parsep}{0pt}}
{\end{itemize}}


%===============================================================================
%                 Sedna Administration Guide: Introduction
%===============================================================================
\begin{document}
\sloppy
\maketitle

\TocAt*{section,subsection,subsubsection}
\TocAt*{subsection,subsubsection}

\tableofcontents
\newpage

\section{Introduction} This guide describes administration of the Sedna XML
Database Management System (Sedna for short). Sedna has the following
components.

\emph{Governor} serves as a ``control center'' of the system. All other
components register at the Governor. The Governor knows which other components
(e.g.\ databases and sessions) are running in the system and controls them.
Other components cannot function properly if the Governor is not running so you
must start it to use Sedna. Besides, Governor is responsible for handling remote
client's requests. For each request it creates a session which keeps the direct
connection with client. Then the client interacts with the system via session.

\emph{Session} keeps the client's settings and allows client to run a sequence
of transactions. The transactions within a session are strictly serial, that is
one transaction ends before next one starts, there is only one active
transaction per session. Transaction provides functionality required for
execution of the client's queries. The query execution consists of several
steps: (1) parsing of the query and translation of the query into its logical
representation, (2) processing of the logical representation by optimizer that
produces the optimal query execution plan, (3) execution of the optimal query
plan and passing results to the client. Transaction is rolled back if error is
occurred during its execution.

\emph{Storage manager} manages a database and provides memory management
functionality to the executor. There is an instance of storage manager for each
database run.

All the components described above are implemented as operating system
processes.

One can run sessions via Sedna client applications (See Sedna Programmer's Guide
how to use Sedna APIs) or via Sedna interactive terminal (see \ref{terminal}).
In both cases, first of all the Governor and storage managers (one for each
databases required) components should be run.

There are a set of command line utilities that allow you to administrate all
aspects of Sedna. These utilities are described in detail in the following
section.


%===============================================================================
%                      Administration via Command Line:
%                         Sedna Directory Structure
%===============================================================================
\section{Administration via Command Line}
\subsection{Sedna Directory Structure}
\label{sec:dir-structure}

The Sedna directory hierarchy is fundamental to obtaining an overall
understanding of the system. Below \verb!SEDNA_INSTALL! refers to the directory
where Sedna is installed. The directory structures are identical on all
supported platforms.

\small{
\begin{verbatim}
SEDNA_INSTALL/bin       command line utilities for managing
                        Sedna and databases, running queries, etc.
SEDNA_INSTALL/cfg       database configuration files named
                        <db_name>_cfg.xml, where <db_name> is
                        the name of the corresponding database
SEDNA_INSTALL/data      database data stored in subdirectories
                        named <db_name>_files, where <db_name>
                        is the name of the corresponding database
SEDNA_INSTALL/lib       database-independent libraries of external
                        functions
SEDNA_INSTALL/data/<db_name>_files/lib
                        database-specific libraries of external
                        functions
SEDNA_INSTALL/doc       Sedna documentation
SEDNA_INSTALL/driver    API drivers for various programming
                        languages
SEDNA_INSTALL/include   include files
SEDNA_INSTALL/etc       Sedna configure sample file
                        sednaconf.xml.sample
SEDNA_INSTALL/examples  example databases and applications for
                        the Sedna APIs
SEDNA_INSTALL/share     Sedna database metadata files (currently
                        only database users and privileges
                        metadata)
\end{verbatim}}

For the installation instructions see the INSTALL file shipped with the
Sedna distribution.


%===============================================================================
%                        Managing Sedna: Running Sedna
%===============================================================================
\subsection{Managing Sedna}
\subsubsection{Running Sedna}
\label{StartSedna}

To start the Sedna server one should start \emph{Governor} by executing the
\verb!se_gov! command. This starts the main server component. After that
Administrator can run databases (as discussed in Section \ref{RunDB}).

The usage of the \verb!se_gov! command is as follows:

\small{
\begin{verbatim}
Usage: se_gov [options]

options:
  --help                   display this help and exit
  -help                    display this help and exit
  -version                 display product version and exit
  -background-mode on/off  start the server in the background mode
                           (default on)
  -port-number <int>       socket listening port
                           (default 5050)
  -el-level level          event logging level (default 3):
                             0 - event logging is off
                             1 - log only fatal errors
                             2 - log all errors/warnings
                             3 - system operational messages
                             4 - log everything (+debug messages)
  -alive-timeout timeout   session keep alive timeout
                           (default 0 - infinite timeout)
  -stack-depth depth       maximum executor stack depth
                           (default 4000)
\end{verbatim}}

\verb!--help! and \verb!-help! options provide the exhaustive information about
the \verb!se_gov! command.

\verb!-version! option allows getting the version of the \emph{Governor}.

\verb!-background-mode! option allows Administrator to start \emph{Governor} in
the background mode.

\verb!-port-number! option specifies the port number on which Sedna listens for
connections from clients.If port number is not specified Governor tries to find
the \verb!sednaconf.xml! file and use port number specified as
\verb!listener_port! parameter (see section \ref{sec:ConfigSedna}, Configuring
Sedna). If it still cannot find port number it will use default 5050 value.

\verb!-el-level! option specifies the event log severity level. For details
refer to the section \ref{sec:event-log}, Event Log.

\verb!-alive-timeout! option specifies number of seconds to wait for the next
request from the some client on the same connection. Session keep alive timeout
can be defined also in the \verb!sednaconf.xml! file. Value provided via command
ine overrides value (if any) defined in the \verb!sednaconf.xml! file.

\verb!-stack-depth! option specifies the maximum safe depth of the session's
physical operations stack. Setting the parameter higher than the actual process
stack may fit will mean that a runaway recursive function can crash an
individual session (and server in the current architecture). The default setting
is 4000, which is unlikely to risk crashes. However, it may be too small to
allow execution of complex functions. Value provided via command line overrides
value (if any) defined in the \verb!sednaconf.xml! file.

To get the information about which Sedna components are run use \verb!se_rc!
command.

The usage of the \verb!se_rc! command is as follows:

\small{
\begin{verbatim}
Usage: se_rc [options]

options:
  --help           display this help and exit
  -help            display this help and exit
  -version         display product version and exit
  -sm-list         display running databases list
\end{verbatim}}


%===============================================================================
%                       Managing Sedna: Stopping Sedna
%===============================================================================
\subsubsection{Stopping Sedna}

To stop Sedna, Administrator should use the \verb!se_stop! command. This command
stops all Sedna components including all databases, which are run. Command has
several optional parameters.

The usage of the \verb!se_stop! command is as follows:

\small{
\begin{verbatim}
Usage: se_stop [options]

options:
  --help       display this help and exit
  -help        display this help and exit
  -version     display product version and exit
  -hard        attempt to roll back transactions immediately
\end{verbatim}}

The first and second options are straightforward, typing these options
Administrator can get the exhaustive information about the \verb!se_stop!
command. Option \verb!-version! allows getting the version of the
\verb!se_stop!.

Note, \verb!se_stop! may cause a rollback of some active transactions in the
system. However, Sedna waits for the last statement in each transaction is
completed. Option \verb!-hard! causes Sedna to attempt to roll back transactions
as soon as possible (except bulk load and update transactions).


%===============================================================================
%                      Managing Sedna: Configuring Sedna
%===============================================================================
\subsubsection{Configuring Sedna}
\label{sec:ConfigSedna}
Sedna configuration parameters are stored in the file \verb!sednaconf.xml!
located in:

\begin{verbatim}
[win:] SEDNA_INSTALL\etc
[nix:] SEDNA_INSTALL/etc
\end{verbatim}

The file is optional. If there is no such file, the default values are used.
The file is an XML document, which satisfies the following DTD:

\small{
\begin{verbatim}
<!ELEMENT sednaconf (sedna_data, os_primitives_id_min_bound,
                     listener_port, ping_port,
                     event_log_level, keep_alive_timeout,
                     session_stack_depth)>
<!ELEMENT sedna_data (#PCDATA)>
<!ELEMENT os_primitives_id_min_bound (#PCDATA)>
<!ELEMENT listener_port (#PCDATA)>
<!ELEMENT ping_port (#PCDATA)>
<!ELEMENT event_log_level (#PCDATA)>
<!ELEMENT keep_alive_timeout (#PCDATA)>
<!ELEMENT session_stack_depth (#PCDATA)>
\end{verbatim}}

The \verb!sedna_data! element contains the path to the directory where database
configuration files (in the \verb!cfg! subdirectory) and databases (in the
\verb!data! subdirectory) are stored. Path must be absolute and exist. The path
by default is \verb!SEDNA_INSTALL!.

The \verb!os_primitives_id_min_bound! element defines the minimal value for
identifiers of OS resources (e.g. semaphores) used in Sedna. In order to run
several Sedna instances on one machine Administrator should configure this
parameter in a such way that identifiers belonging to different Sedna instances
does not intersect. In a standard configuration one Sedna instance acquires a
range of identifiers which length is equal to 150.

The \verb!listener_port! element defines the port number for listening for a
connection requests from a client.

The \verb!ping_port! element defines the port number of Sedna ping server that
is an internal kernel mechanism. It is used to stop all Sedna processes in case
of crash.

The \verb!event_log_level! defines event log severity level (0-4). For full
description see section \ref{sec:event-log}, Event Log.

The \verb!keep_alive_timeout! defines session keep alive timeout, i.e. number of
seconds to wait for the next request from the some client on the same
connection.

The \verb!session_stack_depth! defines the maximum safe depth of the session's
physical operations stack. Setting the parameter higher than the actual process
stack may fit will mean that a runaway recursive function can crash an
individual session (and server in the current architecture). The default setting
is 4000, which is unlikely to risk crashes. However, it may be too small to
allow execution of complex functions.

To tune the Sedna configuration, Administrator should edit the proper elements
of the Sedna configuration file. Note, in order for new configuration to take
effect you must restart Sedna server (\verb!se_gov!).


%===============================================================================
%                    Managing Databases: Creating a Database
%===============================================================================
\subsection{Managing Databases}
\subsubsection{Creating a Database}
\label{CreateDB}

To create a database, use the \verb!se_cdb! command.

Each database consists of the following files, which are located as described
in Section \ref{sec:dir-structure}.

\begin{itemize}
\item The \verb!<db_name>.sedata! file is used for storing persistent XML data,
which are loaded to the database by user using LOAD/UPDATE expressions.
\item The \verb!<db_name>.setmp! file is used for storing temporary XML data.
It stores intermediate results and size may grow during query execution.
\item The \verb!<db_name>.*llog! files are used for storing database logical
log.
\end{itemize}

Besides, for each database there are run-time configuration parameters, which
are stored in the database configuration file (see Section \ref{ConfigDB}).

The usage of the \verb!se_cdb! command is as follows:

\small{
\begin{verbatim}
Usage: se_cdb [options] dbname

options:
  --help                      display this help and exit
  -help                       display this help and exit
  -version                    display product version and exit
  -data-file-max-size Mbs     the max size of data file (in Mb),
                              infinite size by default (0)
  -tmp-file-max-size Mbs      the max size of tmp file (in Mb),
                              infinite size by default (0)
  -data-file-ext-portion Mbs  the data file extending portion
                              size (in Mb), default 100Mb
  -tmp-file-ext-portion Mbs   the tmp file extending portion
                              size (in Mb), default 100Mb
  -data-file-init-size Mbs    the data file initial size (in Mb),
                              default 100Mb
  -tmp-file-init-size Mbs     the tmp file initial size (in Mb),
                              default 100Mb
  -bufs-num N                 the number of buffers in main memory,
                              default 1600 (64Kb per buffer)
  -max-trs-num N              the number of concurrent micro
                              transactions over database
                              (default 10)
  -upd-crt N                  update criterion parameter
                              (fraction of database), default 0.25
  -max-log-files N            maximum log files
                              until log truncate (default: 3)
  -log-file-size Mbs          maximum one log file size (in Mb),
                              (default 100Mb)
  -db-security  level         the level of database security:
                                1) 'off' - none;
                                2) 'authentication' (default);
                                3) 'authorization'

   db_name                    name of the database to be created
\end{verbatim}}

The first and second options are straightforward, typing one of these options
administrator can get the exhaustive information about the \verb!se_cdb!
command.

The third option allows getting the version of the \verb!se_cdb! command.

The \verb!-data-file-max-size! option allows Administrator to set up the maximum
size (in Mb) of the file where the persistent data are stored. The default value
for this option is infinite (0). Note, when database hits this limit it raises
system error (\verb!SE1012!) and shutdowns.

The \verb!-tmp-file-max-size! option allows Administrator to set up the maximum
size (in Mb) of the file where the temporary data are stored. The default value
for this option is infinite (0). Note, when database hits this limit it raises
system error (\verb!SE1012!) and shutdowns.

The \verb!-data-file-ext-portion! option allows Administrator to set up the size
(in Mb) of the portion to which the persistent data file will be extended as
result of executing resize operation. Resize operation is applied to the
persistent data file if its size is less than needed for storing XML documents.
The default value is 100 Mb.

The \verb!-tmp-file-ext-portion! option allows Administrator to set up the size
of the portion to which the temporary data file will be extended as result of
executing resize operation. The resize operation is applied to the temporary
data file if its size is less than needed for storing temporary data
intermediate results of the query, for example). The default value is 100 Mb.

The \verb!-data-file-init-size! option allows Administrator to set up the
initial size (in Mb) of the persistent data file. It is obvious that the initial
size can't be greater than the maximum size of the persistent data file. The
default value is 100 Mb.

The \verb!-tmp-file-init-size! option allows Administrator to set up the initial
size (in Mb) of the temporary data file. It is obvious that the initial size
can't be greater than the maximum size of the temporary data file. The default
value is 100 Mb.

The \verb!-bufs-num! option allows Administrator to set up the number of buffers
in main memory. This number is connected with the size of the main memory, but
the performance of the database depends of this number very much. The default
value is 1600 buffers, which is equal to 100Mb and we do not recommend to use
less number of buffers. The size of one buffer is 64Kb and can't be customized
by Administrator.

The \verb!-max-trs-num! option is the parameter of the Sedna Storage Manager. In
current version of Sedna only default value can be used for this parameter.

The \verb!-upd-crt! option is the parameter of the Sedna Storage Manager to
control snapshot advancement rate. It allows Administrator to specify the
fraction of the database updating of which will result in snapshot advancement.
The default value is 0.25.

The \verb!-max-log-files! option allows Administrator to control how many log
files will be created until Storage Manager will try to truncate logical log by
making a checkpoint. For example, if the value is 3, then after creating
fourth file Storage Manager will try to truncate log. This may significantly
speed up recovery process, since there would be much shorter log to analyze.
Note, however, that checkpoint may be quite a performance-heavy procedure. So
you should not set the value too small. This parameter can be changed later on
every Storage Manager run.

The \verb!-log-file-size! option allows to control size of each log file. When
current log file size becomes greater than the value specified, new log file is
created. With the previous parameter it allows Administrator to find compromise
between frequent checkpoints and faster recovery. This parameter can be set only
on database creation and cannot be changed later.

The \verb!-db-security! option is used to set the level of security in a
database being created. Currently there are 3 possible levels: (1) \verb!off! -
there is no authentication, neither authorization; (2) \verb!authentication! -
only authentication supported (checks user name and password on session open)
(3) \verb!authorization! - authentication and authorization supported (support
for database users and privileges). By default database is created with
\verb!authentication! level. Note, that this parameter is set only once on
database creation and cannot be changed later for the database.

The last parameter \verb!db_name! is the name of the database to be created.
Database name may contain Latin letters, numbers. Also the following special
symbols are allowed \verb!#%&()[]{},.-_=@^`~! (double quote (quote on Windows)
database name parameter to create a database with the name which contains
special symbols). There cannot be two databases with the same name.


%===============================================================================
%                 Managing Databases: Configuring a Database
%===============================================================================
\subsubsection{Configuring a Database}
\label{ConfigDB}

For each database there is a set of run-time configuration parameters, which
store in the database configuration file. The name of the database configuration
file is \verb!<db_name>_cfg.xml! where \verb!<db_name>! is the name of the
database. The location of configuration file is:

\begin{verbatim}
[win:] SEDNA_INSTALL\cfg
[nix:] SEDNA_INSTALL/cfg
\end{verbatim}

The configuration file is an XML document, which satisfies the following DTD:

\small{
\begin{verbatim}
<!ELEMENT db (name, bufs_num, max_trs_num, upd_crt,
              tmp_file_initial_size)>
<!ELEMENT name (#PCDATA)>
<!ELEMENT bufs_num (#PCDATA)>
<!ELEMENT max_trs_num (#PCDATA)>
<!ELEMENT max_log_files (#PCDATA)>
<!ELEMENT upd_crt (#PCDATA)>
<!ELEMENT tmp_file_initial_size (#PCDATA)>
\end{verbatim}}

The root of this XML document is the \verb!db! element, which contains
\verb!name!, \verb!bufs_num!, \verb!max_trs_num!, \verb!max_log_files!,
\verb!upd_crt!, \verb!tmp_file_initial_size! elements. The \verb!name! element
defines the name of the database. The \verb!bufs_num! element defines the number
of buffers in main memory for Storage Manager component. The semantic of
\verb!max_trs_num! element will be defined in future versions of Sedna. The
\verb!max_log_files! element defines the maximum number of log files prior to
truncation attempt. The \verb!upd_crt! element defines the fraction of the
database updating of which will result in snapshot advancement. Finally,
\verb!tmp-file-initial-size! defines initial size of the temporary data file in
megabytes. Once the database is started up, old temporary file is removed and
empty one of this size is initialized. Default value is 100 Mb.
To tune the database run-time configuration, Administrator should edit the
proper elements of the database configuration file. Note, in order for new
configuration to take effect you must restart Storage Manager.


%===============================================================================
%                Managing Databases: Deleting a Database
%===============================================================================
\subsubsection{Deleting a Database}

To drop a database, use the \verb!se_ddb! command. The usage of the
\verb!se_ddb! command is as follows:

\small{
\begin{verbatim}
Usage: se_ddb [options] dbname

options:
  --help            display this help and exit
  -help             display this help and exit
  -version          display product version and exit
   db-name          The name of the database
\end{verbatim}}

The first and second options are straightforward, typing one of these options
Administrator can get the exhaustive information about the \verb!se_ddb!
command.

The third option allows getting the version of the \verb!se_ddb! command.

The last parameter \verb!db_name! is the name of the database to be dropped.

Note that administrator must stop database firstly (see \ref{StopDB}) and then
run \verb!se_ddb! utility.


%===============================================================================
%                Managing Databases: Running a Database
%===============================================================================
\subsubsection{Running a Database}
\label{RunDB}

To run the database with name \verb!db_name! Administrator should use
\verb!se_sm! command. The run-time configuration parameters of the database can
be set by means of \verb!se_sm! command line options. If configuration parameter
is not set via command line option, than the database Storage Manager retrieves
the default value of this parameter from configuration file (see section
\ref{ConfigDB}).

The usage of the \verb!se_sm! command is as follows:

\small{
\begin{verbatim}
Usage: se_sm [options] dbname

options:
  -help                    display this help and exit
  --help                   display this help and exit
  -version                 display product version and exit
  -background-mode on/off  start the server in the background mode
                           (default on)
  -bufs-num N              the number of buffers in main memory,
                           (config file defines default value)
  -max-trs-num N           the number of concurrent micro
                           transactions over database
                           (config file defines default value)
  -upd-crt N               criterion parameter to advance snapshots
                           (config file defines default value)
  -max-log-files N         maximum log files until log truncate
                           (default: 3)
  -tmp-file-init-size Mbs  the tmp file initial size (in Mb),
                           (config file defines default value)
   db-name                 The name of the database
\end{verbatim}}

The first and second options are straightforward, typing this option
Administrator can get the exhaustive information about the \verb!se_sm! command.

The third option allows getting the version of the \verb!se_sm! command. The
\verb!-background-mode! option allows Administrator to start database in the
background mode.

The \verb!-bufs-num! option allows to set up the number of buffers in main
memory. This number is connected with the size of the main memory, but the
performance of the database depends of this number very much. The default value
is retrieved from configuration file. The size of one buffer is 64Kb and can't
be customized.

The \verb!-max-trs-num! option will allow setting up the number of concurrent
micro transactions over database. In current version of Sedna only default value
can be used for this parameter.

The \verb!-upd-crt! option allows Administrator to specify the fraction of the
database updating of which will result in snapshot advancement.

The \verb!-max-log-files! option allows Administrator to control how many log
files will be created until Storage Manager will try to truncate logical log by
making a checkpoint. For example, if the value is 3, then after creating
fourth file Storage Manager will try to truncate log. This may significantly
speed up recovery process, since there would be much shorter log to analyze.
Note, however, that checkpoint may be quite a performance-heavy procedure. So
you should not set the value too small. This parameter can be changed on every
Storage Manager run.

The \verb!-tmp-file-init-size! defines initial size of the temporary data file
in megabytes. Once the database is started up, old temporary file is removed and
empty one of this size is initialized. Default value of this parameter is
retrieved from the database configuration file \ref{ConfigDB}.

The last parameter \verb!dbname! defines the name of the database to be started
up.

Note: before running any database be sure that Sedna Server is strarted up (see
section \ref{StartSedna}).


%===============================================================================
%               Managing Databases: Stopping a Database
%===============================================================================
\subsubsection{Stopping a Database}
\label{StopDB}

To stop the database with name \verb!db_name!, Administrator should use
\verb!se_smsd! command. This command takes one input parameter, which defines
the name of the database to be shut down. The usage of the \verb!se_smsd!
command is as follows:

\small{
\begin{verbatim}
Usage: se_smsd [options] dbname

options:
  --help              display this help and exit
  -help               display this help and exit
  -version            display product version and exit
   db-name            The name of the database
\end{verbatim}}

The first and second options are straightforward, typing these options
Administrator can get the exhaustive information about the \verb!se_smsd!
command. The third option allows getting the version of the \verb!se_smsd!.

Note that \verb!se_smsd! command causes a roll back of all active transactions
executed over \verb!db_name! database.


%===============================================================================
%                             Sedna Terminal
%===============================================================================
\subsection{Sedna Terminal}
\label{terminal}

\verb!se_term! is an interactive terminal to Sedna. It is a C application that
uses Sedna C API (see ``Sedna Programmer's Guide'') to work with Sedna. It
enables you to open a session to one of the Sedna database, type in queries
interactively, issue them to the database and see the query results.
Alternatively, input can be from a file, or a single query can be passed for
execution as a command line parameter. In addition, it provides a number of
meta-commands.

The usage of the \verb!se_term! is as follows:

\small{
\begin{verbatim}
Usage: se_term [options] dbname

options:
  -help               display this help and exit
  --help              display this help and exit
  -version            display product version and exit
  -file filename      file with an XQuery query
  -output filename    output file (default stdout)
  -query "query"      XQuery query to execute
  -echo on/off        display se_term output
                      (default: on for interactive mode,
                                off for batch mode)
  -show-time on/off   show time of the latest query execution
                      (default off)
  -debug on/off       execute statements in debug mode
                      (default off)
  -host host          hostname of the machine with Sedna running
                      (default localhost)

  -port-number port   socket listening port (default 5050)
  -name name          user name
  -pswd password      user password
   db-name            database name
\end{verbatim}}

\verb!-help! or \verb!--help! provides the exhaustive information about the
\verb!se_term!.

\verb!-version! option allows getting the version of the \verb!se_term!.

\verb!-file! option specifies the name of the file that contains any number of
queries and meta-commands. \verb!se_term! uses this file as the source of
queries and meta-commands instead of reading them interactively. \verb!se_term!
executes the queries and meta-commands from file consequently and exit. XQuery
and XUpdate statements must be delimited with ampersand symbol at the end.

For example, the following script turns off autocommit mode, then executes a
number of XQuery statements and finally commits transaction explicitly:

\small{
\begin{verbatim}
\nac
CREATE DOCUMENT "test"&
UPDATE INSERT <test>{"test"}</test> INTO fn:doc("test")&
fn:doc("test")&
\commit
\end{verbatim}}

\verb!-output! option specifies a filename of a file to redirect all
\verb!se_term! output to.

\verb!-query! option specifies the query to execute.

\verb!-echo! option specifies if \verb!se_term! output needs to be displayed or
not. If the \verb!echo! is on, the output is displayed, if the \verb!echo! is
off, the output is not displayed. By default, the \verb!echo! option is set to
on when using \verb!se_term! in an interactive mode, and the \verb!echo! option
is set to off when using \verb!se_term! in a batch mode (running
queries/commands from file).

\verb!-show-time! option allows getting the time of the latest query execution.
If used with the \verb!-query! option provides the time of the specified query
execution; if used with the \verb!-file! option provides the time of the
execution of the last query in file.

\verb!-debug! option specifies session debug mode. If \verb!-debug! option is
on, statements of this session are executed in a debug mode. If \verb!-debug!
option is off, statement of this session are executed in normal (not debug)
mode. For details see "Debug Facilities" section of the Sedna Programmer's
Guide.

\verb!-host! option specifies the name of the machine with Sedna DBMS running.
If not used, default value \verb!localhost! is used.

\verb!-port! option specifies the TCP port on which Sedna server is listening to
client applications. If omitted, port number 5050 is used by default.

\verb!-name! and \verb!-password! options specify the user name and the
password. If omitted, connects as a predefined user \verb!SYSTEM! with password
\verb!MANAGER!.

\verb!dbname! is a required argument. It specifies the name of the database to
connect to.

When \verb!se_term! is used in the interactive mode it takes in query/update
statements and meta-commands from stdin. To execute a query/update statement
type in a statement and use ampersand and line feed to terminate it
(\verb!&'\n'! terminates the query). An end of line does not terminate the
query, thus queries can be spread over several lines for clarity. Alternatively,
\verb!se_term! can be used in a batch mode, then query/update statements and
meta-commands are taken from the input file specified by the \verb!-file!
option. By default, \verb!se_term! session is run in the \emph{autocommit mode},
that is, each statement is run in a separate transaction. To switch to a
\emph{manual-commit mode} use \verb!unset! meta-command (\verb!se_term!
meta-commands are described below in this section). If the statement executed
successfully, the results are displayed on the screen.

\verb!se_term! returns 0 to the shell if it finished normally, 1 if a fatal
error of its own (out of memory, file not found) occurs, 2 if the connection to
the Sedna server went bad and the session is not interactive, and 3 if a
statement or a command failed and the variable \verb!ON_ERROR_STOP! was set
(\verb!se_term! internal variables are described below in this section).

Meta-commands are commands for \verb!se_term! that processed by the
\verb!se_term! itself. A meta-command begins with a backslash and that differs
it from the query/update statements. Thus, the format of \verb!se_term!
meta-command is the backslash, followed immediately by a command, with no
ampersand at the end. \verb!se_term! takes in the following meta-commands:

\small{
\begin{verbatim}
 \? - for help on internal slash commands
 \commit - to commit transaction
 \rollback - to rollback transaction
 \showtime - to show the time of the latest query execution
 \set - to set the terminal internal variable
 \unset - to unset the terminal internal variable
 \quit, \q - to close session and quit the Sedna Terminal
\end{verbatim}}

\verb!set! and \verb!unset! meta-commands are used for managing \verb!se_term!
internal variables. Notice, that there are aliases for some commands. So you can
write \verb!\ac! instead of \verb!\set AUTOCOMMIT!. There are following
\verb!se_term! internal variables:

\verb!AUTOCOMMIT (\ac for set, \nac for unset)! - when set, autocommit mode is
on. When unset manual-commit mode is on. \verb!AUTOCOMMIT! is set by default.

\verb!ON_ERROR_STOP! - when set, \verb!se_term! returns with the code 3 when
statement or meta-command fails. When unset \verb!se_term! processing continues,
unless it is the connection failure.

\verb!DEBUG! - when set, session debug mode is on. When unset, session debug
mode is off. See "Debug Facilities" section of the Sedna Programmer's Guide for
details.

\verb!TRANSACTION_READ_ONLY! \verb!(\ro for set, \upd for unset)! - transactions
are run as \verb!READ-ONLY! when set. When unset, transactions are run as
\verb!UPDATE!-transactions. By default transactions are run as \verb!UPDATE! -
transactions.

\verb!LOG_LESS_MODE (\ll for set, \fl for unset)! - when set, every following
bulkload will be less logged and checkpoint will be made on every commit. When
unset, every following bulkload will be fully logged. By default transactions
are run in full log mode. This option should be used with care (see details
about \verb!SEDNA_LOG_AMOUNT! connection attribute in ``Sedna Programmer's
guide'').

\verb!QUERY_TIMEOUT=<time in seconds>! - when set, every query execution will be
dropped on server if it lasts longer than timeout set. By default there is no
any timeout set (query is executed as long as needed).

\verb!set?! - provides help on \verb!se_term! internal variables.


%===============================================================================
%                           Backup: Export Utility
%===============================================================================
\subsection{Backup and Restore}

As any database system that contains valuable data, Sedna databases should be
backed up regularly. In this section we present different approaches to back up
Sedna data and the process of migration between different Sedna releases.

\subsubsection{Export/Import Utility}

The purpose of \verb!se_exp! utility is to provide functionality of
exporting/importing data. The idea behind the \verb!se_exp! method is to
generate a set of XML files and XQuery scripts to restore the database in the
same state as it was at the time of the exporting.

Note that \verb!se_exp! is a regular Sedna client application. This means that
you can export data from any remote host that has access to Sedna. The
requirement is that \verb!se_exp! should operate with special permissions. In
particular, it must have read access to all documents in the database including
system metadata. This means that you should run it as a user which has DBA role.

The usage of \verb!se_exp! utility is:

\small{
\begin{verbatim}
se_exp [options] command dbname path
options:
  -help               display this help and exit
  --help              display this help and exit
  -version            display product version and exit
  -verbose on/off     verbose output (default off)
  -host host          hostname of the machine with Sedna running
                      (default localhost)
  -port-number port   socket listening port  (default 5050)
  -name name          user name
  -pswd password      user password
   command            export | restore | import
   db-name            database name
   path               path to exported/imported data
\end{verbatim}}

There are three commands to manipulate data with \verb!se_exp!. They are
\verb!export!, \verb!restore! and \verb!import!. Below we describe each of these
commands in details.


\subsubsection*{Export}

The purpose of \verb!export! command is to export data from specified database.
The basic usage of this command is:

\begin{verbatim}
se_exp export dbname path
\end{verbatim}

The parameter \verb!dbname! specifies the database in Sedna to export data from.

The \verb!path! parameter specifies the directory to store files with exported
data.

\begin{note}
If the directory to which \verb!path! refers contains files that have the same
names as the files created by \verb!se_exp! they will be replaced.
\end{note}

While export process \verb!se_exp! generates a set of XML files and a set of
XQuery scripts to recreate the state of database. For each XML document in the
database including XML documents in collections \verb!se_exp! generates an XML
file. Some XML files with system metadata are also generated. Note that the
security metadata is exported in insecure way (the file contains unencrypted
user names and passwords).

Exported data created by \verb!se_exp! is transaction consistent, that is,
updates to the database while \verb!se_exp! is running will not be in the
exported data.

To specify which database server \verb!se_exp! should contact, use the command
line options \verb!-host host!. The default host is the local host. As Sedna
client application, \verb!se_exp! requires user name and password to connect to
the database. You can either specify them with \verb!-name! and \verb!-pswd!
options or type user name and password in the dialog while running
\verb!se_exp!.


\subsubsection*{Restore}

The \verb!restore! command restores data created by the export command into the
empty database. The restore command is intended for migration between different
releases of Sedna and for back up of your data in XML format. The basic usage of
this command is:

\begin{verbatim}
se_exp restore dbname path
\end{verbatim}

The parameter \verb!dbname! specifies the database in Sedna to restore data
into. The \verb!path! parameter specifies the directory with data to restore.

The database \verb!dbname! will not be created by this command. It is required
that the target database already exist and run before starting the restore
process. You must create it yourself with the help of \verb!se_cdb! command and
start it with \verb!se_sm! command. It is also required that the target database
is empty, i.e. it doesn't contain any data or any users or roles except the
default one.


\subsubsection*{Import}

The \verb!import! command imports data created by the export command into an
existing database which may be not empty. The basic usage of this command is:

\begin{verbatim}
se_exp import dbname path
\end{verbatim}

The parameter \verb!dbname! specifies the database in Sedna to import data to.
The \verb!path! parameter specifies the directory with data to import.

The database \verb!dbname! will not be created by this command. It is required
that the target database already exist and run before starting the import
process.

You can import data into database which contains some data and has some security
politics. The only restriction is that there should not be any conflicts in the
names of XML documents, collections or indices.

The main difference between \verb!restore! and \verb!import! command is that
\verb!import! command doesn't import any security information. All data is
imported by the user who run the \verb!se_exp! utility, i.e. by the user with
the name and password specified with \verb!se_exp! parameters. Of cause this
user should have enough rights to create collections, load documents and create
indices.



%===============================================================================
%                          Backup: File System Backup
%===============================================================================
\subsubsection{File system level backup}
\label{sec:file-system-level-backup}

An alternative strategy to backup a database is to directly copy the directories
that Sedna uses to store the data of the database. Read Section
\ref{sec:dir-structure} to find out where Sedna stores databases. You can use
whatever method you prefer for doing usual file system backups. To restore a
database, copy the corresponding backup directory to the location where Sedna
stores databases.

There is a requirement is that the target database must be stopped in order to
get a usable backup. Half-way measures such as disallowing all connections will
not work.

Note that a file system backup will not necessarily be smaller than an back up
via export. On the contrary, it will most likely be larger.

\begin{note}
The database directory copied to different machine or different version of the
same operating system might not work properly. If you want to restore a database
on another machine or OS installation use \verb!se_exp! utility instead.
\end{note}


%===============================================================================
%                            Backup: Hot Backup
%===============================================================================
\subsubsection{Hot Backup}

Another alternative is to backup a database while it is still running. Such
procedure is called hot backup. The purpose is to create a consistent backup
copy while users are still performing some requests. This copy can then be
restored by copying corresponding backup directory to the directory where Sedna
stores databases. Such hot backups can be done in incremental mode, which allows
more efficient archiving of database changes.

In a nutshell, when hot backup is called Sedna makes copies of all database
files necessary to restore consistent database state in case of failure. The
main difference between file system level backup (described in Section
\ref{sec:file-system-level-backup}) and hot backup is that the target database
has not to be stopped. As a tradeoff, restoration from hot backup copy may be a
slower process, depending on the recency of the copy. Note, that hot backup copy
guarantees durability of all transactions that had been committed at the moment
of starting hot backup process.

To make a hot backup you must use provided \verb!se_hb! utility. The usage is as
follows:

\small{
\begin{verbatim}
Usage: se_hb [options] dbname path

options:
  -help                        display this help and exit
  --help                       display this help and exit
  -checkpoint                  make checkpoint before backup
  -time-dir                    create timestamp-subdir
  -make-dir                    create directory if it doesn't exist
  -incr-mode <increment_mode>  increment mode (start, add, stop)
  -port port-number            port number to connect to Governor

  dbname                       the name of the database
   path                        the name of the backup directory
\end{verbatim}}

The parameter \verb!dbname! specifies the database to backup.

The parameter \verb!path! specifies the directory to store files of the hot
backup.

You can use \verb!-checkpoint! option to make sure that checkpoint is made
before hot backup process is started. This may make restoration process faster,
since checkpoint fixates consistent state of the database and this state will be
reflected in the hot backup copy. But at the same time backup process may take
more time, depending on the user activity at the time of the backup.

If you want the destination directory (specified as \verb!path! in command line)
to be created, you must specify \verb!-make-dir! option.

Note that \verb!se_hb! may overwrite some of the previous backup files if you
specify nonempty destination directory. So, if you make two consequent hot
backups of the same database in the same directory, the older backup will be
lost. \verb!-time-dir! option prevents this by creating subdirectory named with
the current date-time within \verb!path! directory. In this case the destination
of hot backup copy will be:
\verb!<path>/backup-<dbname>-<current date>-<current time>/!.
It is recommended that you use the \verb!-time-dir! option or provide a
directory free of the previous backups.

You can specify port number to connect to governor through \verb!-port! option.
If port number is not specified in command line, hot backup process tries to
find \verb!sednaconf.xml! file and use port number specified as
\verb!listener_port! parameter. If it still cannot find port number, it will try
to use default 5050 value.

\begin{note}
You must run \verb!se_hb! utility on the same machine as the target database is
running. It will try to connect to the target database through the specified
port to the \verb!localhost!.
\end{note}

Incremental hot backup and corresponding \verb!-incr-mode! option will be
explained below.


\subsubsection*{Incremental Hot Backup}

Let us assume that you have made hot backup using command like this
\verb!se_hb mydb /backup!. If later you want to make another one to be sure
updated data will be reflected in the backup copy, you can issue the same
command. However, if the amount of changes is small, it is desirable to copy
only this changes without making copy of the entire database again. This is
where incremental mode becomes useful.

First of all, you must create a primary copy, which is essentially the copy of
the entire database. You can do it by specifying \verb!-incr-mode start! in
command line (for example, \verb!se_hb -incr-mode start mydb /backup!). Then,
when you need to make subsequent hot backups of the same database, you can
specify \verb!-incr-mode add! in command line (for example,
\verb!se_hb -incr-mode add mydb /backup!). If the amount of changes is small,
such backup process will take much less time.

Note, however, that when you use \verb!-incr-mode start! option it will switch
the original (active) database in incremental mode. This means it will start to
store more files to allow ``\verb!-incr-mode add!'' backups. In this case
original database can grow in size more rapidly in case it is updated. To switch
off incremental mode you must specify \verb!-incr-mode stop! in command line.
This command allows the original database to drop unnecessary files, but it also
makes ``\verb!-incr-mode add!'' option impossible. Thus, to start another
incremental backup process you must repeat the whole process all over again
(\verb!-incr-mode start! call and additional \verb!-incr-mode add! calls when
needed). Note that \verb!se_hb! with \verb!-incr-mode stop! option does not make
any additional hot backup copies, it just switches off incremental mode.

\begin{note}
The database switches off incremental mode when new nonincremental (without
\verb!-incr-mode! option) hot backup is created. This is similar to the
\verb!-incr-mode stop! option, only in this case hot backup copy of the entire
database is created.
\end{note}

If you make new primary copy (i.e. with the \verb!-incr-mode! \verb!start!
option specified) while database is still in incremental mode,
\verb!-incr-mode add! option will archive increments valid only for this last
primary copy. It is recommended that you periodically make new primary copy with
\verb!-incr-mode start! option. Unless, of course, the database is rarely
updated.

With incremental hot backup you have two options: you can archive all backups in
the same directory or in different directories. If you archive all backups in
the same directory (as in \verb!/backup! in our example above) you can restore
only state corresponding to the last incremental backup. On the other hand,
storing incremental backups in different directories makes possible some kind of
point-in-time recovery, i.e. you can restore state corresponding to any of the
incremental backups. For the details see the next section.

\subsubsection*{Restore from Hot Backup Copy}

\begin{note}
Since hot backup is made on file system level basis, the same note as in the
``File system level backup'' section applies here too, i.e. recovery of the hot
backup copy on different machine or different version of the same operating
system cannot be guaranteed.
\end{note}

\begin{note}
Recovery of hot backup copy on different release of Sedna cannot be guaranteed.
See Section \ref{sec:release-migr} for further details.
\end{note}

To restore the backed up database you must copy saved files to the directory
where Sedna stores database files. For the information about Sedna directory
structure read Section \ref{sec:dir-structure}. For example, let us assume that
hot backup have been made in \verb!/backup! directory, and Sedna stores its
files in \verb!SEDNA_INSTALL! directory. In this case you can find \verb!cfg!
and \verb!data! subdirectories in \verb!/backup! directory. To restore database
you must copy this directories in \verb!SEDNA_INSTALL! directory. Note, that you
should remove \verb!SEDNA_INSTALL/data/<dbname>_files/! (where \verb!<dbname>!
is the name of the backed up database) directory before you copy backup files,
since old files may interfere with restoration process. After you copy backup
files in the corresponding directories, you can use \verb!se_sm! command to
start the database. When you do it for the first time SM runs recovery process
to restore database state corresponding to the moment hot backup took place. It
can take some time depending on the recency of the copy.

For the incremental backed up database the process may be different. If you have
made all backups (primary copy and additional ``\verb!-incr-mode add!'' copies)
in the same directory the process is the same. However, if you have some of the
``\verb!-incr-mode add!'' copies in the different directories you must copy
files from all this directories in order corresponding hot backups were taken to
fully restore database state. This makes possible restoration of older state of
the database. For example, let us assume that primary copy is stored in
\verb!/backup/p! directory and additional \verb!-incr-mode add! copies are
stored in \verb!/backup/1! and \verb!/backup/2! directories in the order of
creation. Then you can restore database to the state corresponding to any of the
hot backups by copying only those directories that you need. For example, by
copying \verb!/backup/p and !\verb!/backup/1! directories you can restore
database state corresponding to the moment when \verb!-incr-mode add /backup/1!
was made. Of course, you cannot ``skip'' directories. For example, copying only
\verb!/backup/p! and \verb!/backup/2! would not be possible, since in this case
\verb!/backup/1! is also needed.

%===============================================================================
%                         Migration Between Releases
%===============================================================================
\subsubsection{Migration Between Releases}
\label{sec:release-migr}

In this section we discuss how to migrate your data from one release of Sedna to
another. As the internal data storage format is subject to change between
different releases of Sedna it is a frequently required task to accurately
migrate data.

It is recommended that you use \verb!se_exp! utility to pass through this
problem. The process consists of four steps.

\begin{enumerate}
\item Run \verb!se_exp! utility with \verb!export! command to export your data
to some directory on the filesystem.
\item Shut down the old version of Sedna database server and remove it from your
system.
\item Install and run a new version of Sedna, create the database with
\verb!se_cdb! and start it with \verb!se_sm!. Make sure that no transaction has
been run with the new database before restoring the data.
\item Run \verb!se_exp! utility with \verb!restore! command to restore data into
the new database.
\end{enumerate}

%===============================================================================
%                                Event Log
%===============================================================================
\subsection{Event Log}
\label{sec:event-log}

Sedna keeps track of all important events that happen during its functioning.
This section describes where log files are located and how to configure Sedna
logging capabilities.

\subsubsection{Files}
Log files are located in the same folder where Sedna stores databases (see
section \ref{sec:dir-structure}, Sedna Directory Structure). By default it means
that you can find event log files in \verb!SEDNA_INSTALL/data!. Latest log file
is named \verb!event.log!. Sedna also stores old event log files which are named
\verb!event-{data}.log!.

\subsubsection{Severity Level Configuration}
Sedna provides a flexible way to set severity level of the logging output. It
can be done either through \verb!se_gov! command line parameter:

\begin{alltt}
se_gov -el-level \emph{number}
\end{alltt}

or defined in \verb!sednaconf.xml! (see section \ref{sec:ConfigSedna},
Configuring Sedna):

\begin{alltt}
<event_log_level>\emph{number}</event_log_level>
\end{alltt}

The following severity levels are available:
\\
\begin{tabular}{|l|l|}
\hline
\hline
Level    & Description                                                        \\
\hline
\hline
0        & Event logging off.                                                 \\
\hline
1        & Logs fatal errors that caused all database sessions to abort.      \\
\hline
2        & Logs all errors and warnings.                                      \\
\hline
3        & Logs detailed information (queries, IO statistics, etc).           \\
\hline
4        & Logs successively-more-detailed information for use by developers. \\
\hline
\end{tabular}

%===============================================================================
%                          Command Line Examples
%===============================================================================
\subsection{The Examples of Using Command Line Utilities}

In this section we present several examples of using the utilities described in
the previous sections. These examples demonstrate how to create the database,
run the Sedna server and the database, load documents into the database, run the
query over the database, and finally stop Sedna.

To run Sedna, type the following command:

\begin{verbatim}
se_gov
\end{verbatim}

For creating a database named \verb!xmark! with default settings use the
following command:

\begin{verbatim}
se_cdb xmark
\end{verbatim}

After execution of these commands the created database can be started up. To do
it, type the following command:

\begin{verbatim}
se_sm xmark
\end{verbatim}

The \emph{xmark} database is ready for serving user queries. Below we show how
to load an XML document to the xmark database and how to write a queries over
this document.

To load the XML document into the \emph{xmark} database, pass the following
steps:

\begin{enumerate}
\item Create the following file with name \verb!load.xquery!:

\begin{verbatim}
LOAD "<path>" "<name>"
\end{verbatim}

The \verb!<path>! is the path to the XML document to be loaded to the
\emph{xmark} database. The \verb!<name>! is the name of this document in the
\emph{xmark} database.

\item Run the command:

\begin{verbatim}
se_term -file load.xquery xmark
\end{verbatim}

\item Create the following file with name \verb!query1.xquery!:

\begin{verbatim}
doc("<name>")/*
\end{verbatim}

Where \verb!<name>! is the name of the loaded document in the xmark database.

\item Type the command and enter it:

\begin{verbatim}
se_term -file query1.xquery xmark
\end{verbatim}
\end{enumerate}

To stop Sedna and all its components and databases, type and enter the following
command:

\begin{verbatim}
se_stop
\end{verbatim}

%===============================================================================
%                           Authorization System
%===============================================================================
\section{User-Based Authorization System}

This chapter describes how to create and manage users and introduces the
privilege system.

When created with \verb!-db-security authorization! option, a database contains
a set of database users. Those users are separate from the users managed by the
operating system on which Sedna runs. The primary function of the Sedna
privilege system is to authenticate a user and to associate that user with
privileges on a database object such as \verb!DROP!, \verb!CREATE! or
\verb!QUERY!. Users own database objects (for examples, document) and can
assign privileges on those objects to other users.


%===============================================================================
%                        Authorization: Database Users
%===============================================================================
\subsection{Database Users}

Database user names are global across a database (and not per all Sedna
databases). \emph{Database users} interact with \emph{database objects}. Every
database object has its \emph{owner} - the user that created it. Every user and
\emph{role} (we will discuss roles in the Section \ref{roles}) has its
\emph{creator}.

In order to bootstrap the database, a freshly created database always contains
one predefined DBA user with name \verb!"SYSTEM"! and password \verb!"MANAGER"!.
To start your work with the database, you first have to connect as this initial
user, then you can create more users and change default password (if you care
for preventing unauthorized access to your database).

There are following kinds of Sedna database objects:
\begin{citemize}
\item Standalone document
\item Collection of documents
\item Value based index
\item Full-text index
\item Module
\item Trigger
\item Metadata document
\end{citemize}

There are two types of Sedna database users:

\begin{citemize}
\item Database administrator (DBA user). Formally, DBA user is a user that has
the "DBA" role.
\item Ordinary user (below we call "user")
\end{citemize}

DBA user:
\begin{citemize}
\item has all possible privileges on any object in the database;
\item can remove any object in the database;
\item can remove any user of the database;
\item can grant/revoke any privilege to/from any user of the database;
\item can grant "DBA" role to a user, thus making that user also a DBA user
(not recommended, as the database with multiple DBA users is hard to
administrate). Any DBA user can also revoke the "DBA" role from any DBA user.
\end{citemize}

An ordinary user:
\begin{citemize}
\item can act according to the privileges that he has;
\item can grant and revoke any privileges on the database object that he owns
to any user;
\item can remove database objects that he owns and drop users that he has
created.
\end{citemize}

Every user has its name and password.

To create a user use \verb!CREATE USER! statement:

\begin{verbatim}
CREATE USER "user-name" WITH PASSWORD "user-password"
\end{verbatim}

For example, the following statement:

\begin{verbatim}
CREATE USER "Alice" WITH PASSWORD "mypass"
\end{verbatim}

creates user \emph{Alice} identified with \emph{mypass} password.

To remove an existing user, use \verb!DROP USER! statement:

\begin{verbatim}
DROP USER "user-name"
\end{verbatim}

For example, the following statement removes user \emph{Alice}:

\begin{verbatim}
DROP USER "Alice"
\end{verbatim}

A user can drop only a user he has created. DBA user can drop any user of the
database.

To change user password use \verb!ALTER USER! statement. A user can change a
password for himself or for a user he has created. DBA user can change a
password for any user of the database.

\begin{verbatim}
ALTER USER "user-name" WITH PASSWORD "new-password"
\end{verbatim}

This statement changes the password of the user \emph{user-name} to
the \emph{new-password}.


%===============================================================================
%                    Authorization: Privileges and Roles
%===============================================================================
\subsection{Privileges and Roles}
\label{roles}

When a database object is created, it is assigned an owner. The owner is the
user that executed the creation statement. By default, only an owner and DBA
user can do anything with the database object. In order to allow other users
to use it, \emph{privileges} must be granted.

There are several possible privileges:

\begin{citemize}
\item CREATE-USER - create new user
\item CREATE-DOCUMENT - create new document
\item CREATE-COLLECTION - create new collection
\item CREATE-INDEX - create new value-based index
\item CREATE-FT-INDEX - create new full-text index
\item CREATE-TRIGGER - create new trigger
\item LOAD-MODULE - load new module into database
\item LOAD - load new document either into database or collection
\item DROP - drop index, documen, collection, user, role
\item QUERY - query document, collection or database
\item INSERT - perform update insert statements
\item DELETE - perform update delete statements
\item RENAME - perform update rename statements
\item RETRIEVE-METADATA - retrieve metadata
\end{citemize}

Privileges are assigned to database objects or to the whole database. The table
below for every privilege lists all kinds of database objects or the whole
database it can be assigned to.

\medskip

\begin{tabular}{|l|l|}
\hline
\hline
Privilege            & can be assigned to             \\
\hline
\hline
CREATE-USER          & DATABASE                       \\
\hline
CREATE-DOCUMENT      & DATABASE, COLLECTION           \\
\hline
CREATE-COLLECTION    & DATABASE                       \\
\hline
CREATE-INDEX         & DOCUMENT,COLLECTION            \\
\hline
CREATE-FT-INDEX      & DOCUMENT,COLLECTION            \\
\hline
CREATE-TRIGGER       & DOCUMENT,COLLECTION            \\
\hline
LOAD-MODULE          & DATABASE                       \\
\hline
RETRIEVE-METADATA    & DATABASE                       \\
\hline
LOAD                 & DATABASE, COLLECTION           \\
\hline
DROP                 & DOCUMENT, COLLECTION, INDEX,   \\
                     & MODULE, TRIGGER, FT-INDEX      \\
\hline
QUERY                & DOCUMENT, COLLECTION           \\
\hline
INSERT               & DOCUMENT, COLLECTION           \\
\hline
DELETE               & DOCUMENT, COLLECTION           \\
\hline
RENAME               & DOCUMENT, COLLECTION           \\
\hline
\end{tabular}

\medskip

Plus there is the key word \verb!ALL! that denotes all possible privilege that
can be granted to the specified object.

Note, \verb!DROP! privilege on collection is granted to a user means that the
user can drop any document in this collection (there is no such kind of database
object as 'document in collection'). \verb!LOAD! privilege on the database
allows user to load standalone documents; \verb!LOAD! privilege on a collection
allows user to load documents into the collection.

\emph{Role} is a named group of related privileges. Roles provide easy and
controlled way to manage privileges. To create role \verb!CREATE ROLE! statement
is used:

\begin{verbatim}
CREATE ROLE "role-name"
\end{verbatim}

This statement creates role with name \emph{role-name}. When created role does
not contain any privileges. It is not recommended to created a role with a name
of an existing user (in this case privileges will be granted both to this user
and to this role).

It is not allowed to use "DBA" and "PUBLIC" for naming roles as they are
reserved by the system.

To destroy a role, use \verb!DROP ROLE!:

\begin{verbatim}
DROP ROLE "role-name"
\end{verbatim}

Role is automatically revoked from any user it was granted to.


%===============================================================================
%                    Authorization: Granting Privileges
%===============================================================================
\subsection{Granting privileges}

Privileges are granted to users so that users can access and operate with
database objects or to process some tasks with a database.

A user can receive a privilege in two different ways: privileges can be granted
to user explicitly; or privileges can be granted to role, and the role can be
granted to one or more users. Roles allow easier and better management of
privileges, thus privileges are normally granted to roles and not to specific
users. Roles can be granted both to users and to roles.

A user who has granted a privilege or a role is \emph{grantor} of this privilege
(role).

To grant a privilege on a database object you must be an owner of this object or
DBA user. To grant one or more privileges on a database object to one or more
users or roles use:

\begin{verbatim}
GRANT "privilege" | ALL
ON [DOCUMENT|COLLECTION] "database-object-name"
TO "user-name|role-name" | PUBLIC
\end{verbatim}

For example, the following statement:

\begin{verbatim}
GRANT "QUERY" ON DOCUMENT "auction" TO "Alice"
\end{verbatim}

grants \verb!QUERY! privilege on document \emph{auction} to the user with name
\emph{Alice}. She will be able to perform any XQuery queries on this document.

The key word \verb!PUBLIC! is used when the privileges are to be granted to all
users, including those that may be created later. \verb!PUBLIC! may be thought
of as an implicitly defined role that every user has.

If the kind of the database object (\verb!DOCUMENT! or \verb!COLLECTION!) is not
specified, database object is considered to be a document.

To grant a privilege on a database you must be DBA user. To grant one or more
privileges on a database to one or more users or roles use:

\begin{verbatim}
GRANT "privilege" | ALL
ON DATABASE
TO "user-name|role-name" | PUBLIC
\end{verbatim}

For example, the following statement:

\begin{verbatim}
GRANT "LOAD" ON DATABASE TO "Alice"
\end{verbatim}

allows \emph{Alice} to perform bulk load into the database.

To grant a role you must be allowed to grant every privilege of the role: to be
owner of every database object of privileges of the role, or to be DBA user. To
grant a role to another role means grantee will add grantor's privileges to its
own set of privileges. To grant a role to one or more users or roles use:

\begin{verbatim}
GRANT "role-name"
TO "user-name|role_name" | PUBLIC
\end{verbatim}


%===============================================================================
%                   Authorization: Revoking Privileges
%===============================================================================
\subsection{Revoking privileges}

Privileges or roles can be revoked from the user. Roles cannot be revoked from
roles, however. Only grantor of the privilege (role) or DBA user can revoke
privilege (role).

\verb!REVOKE! statements are similar to \verb!GRANT! statements.

To revoke one or more privileges on a database object from one or more users or
roles use:

\begin{verbatim}
REVOKE "privilege" | ALL
ON [DOCUMENT|COLLECTION] "database-object-name"
FROM "user-name|role-name" | PUBLIC
\end{verbatim}

If the kind of the database object (\verb!DOCUMENT! or \verb!COLLECTION!) is not
specified, database object is considered to be a document.

To revoke one or more privileges on a database from one or more users or roles
use:

\begin{verbatim}
REVOKE "privilege" | ALL
ON DATABASE
FROM "user-name|role-name" | PUBLIC
\end{verbatim}

To revoke a role from one or more users use:

\begin{verbatim}
REVOKE "role-name"
FROM "user-name" | PUBLIC
\end{verbatim}

As mentioned above DBA user is a user that has a "DBA" role. Thus, "DBA" is a
reserved name for a role: a role with name "DBA" can not be created, privileges
or roles cannot be granted to "DBA" role.

A DBA user can grant "DBA" role to another user, thus making that user also a
DBA user. This is not recommended, as multiple powerful users of a database can
lead to hard database administration, and can cause insecure usage of the
database and database objects.


%===============================================================================
%                           Client Authentication
%===============================================================================
\subsection{Client Authentication}

When a client application connects to the database server, it specifies which
Sedna database user name it wants to connect as. User name determines access
privileges to database objects, therefore, client authentication is used to
restrict which database users can connect.

\emph{Authentication} is the process by which the database server establishes
the identity of the client and determines whether the client application (or the
user who runs the client application) is permitted to connect with the user name
that was requested.

When created with \verb!-db-security authorization! or
\verb!-db-security authentication! option, database will check the user password
on session open.

Currently, Sedna uses password authentication: client application that connects
to the database must specify user name and user password (in \verb!se_term!
utility use \verb!-pswd! and \verb!-name! options; for client application that
works through Sedna API see "Sedna Programmer's Guide"). Authentication process
consists in checking the password correctness.

A newly created database always contains one predefined DBA user with name
\verb!"SYSTEM"! and password \verb!"MANAGER"!. To start your work with the
database, you first have to connect as this initial user.


%===============================================================================
%                              Localization
%===============================================================================
\section{Localization}

In the current version of Sedna all character data are stored internally in
UTF-8. All input queries and XML documents must be encoded in UTF-8. All output
is also encoded in UTF-8.

\end{document}

