\documentclass{article}
\usepackage{a4wide}
\usepackage{bbding}
\usepackage{boxedminipage}


\begin{document}

\section{C{\small ++} Exceptions}

\begin{itemize}
\item Use exception classes defined in appendix \ref{sec:exnclasses}.
\item Never use \verb|catch(...)|, use \verb|catch(ANY_SE_EXCEPTION)| instead\footnote{We have code-generation problem with \texttt{cl} compiler older than v14.}.
\item When adopting external library make sure that the library either doesn't use exceptions at all or is throwing exceptions derived from \texttt{std::exception}\footnote{\texttt{ANY\_SE\_EXCEPTION} is coded with assumption that \emph{every} exception object is derived from \texttt{std::exception}.}.
\end{itemize}

\newpage
\appendix
\section[Sedna Exception Classes]{Sedna Exception Classes\footnote{originating from notes in \texttt{kernel/common/errdbg/exceptions.h} by AF}}
\label{sec:exnclasses}

The following primary exceptions are defined for Sedna (later referred as 
``the system") -- see fig.~\ref{fig:exceptions}.

\begin{figure}[ht]
\centering
\begin{boxedminipage}{4in}
{\Large SednaException}
	\begin{list}{\ArrowBoldDownRight}{}
	\item\textbf{SednaSystemException}
		\smallskip
		\begin{list}{\ArrowBoldRightStrobe}{}
		\item SednaSystemEnvException
		\end{list}
		\smallskip
	\item\textbf{SednaUserException}
		\smallskip
		\begin{list}{\ArrowBoldRightStrobe}{}
		\item SednaUserExceptionFnError
		\item SednaUserEnvException
		\item SednaUserSoftException
		\item SednaXQueryException
		\end{list}
	\end{list}
\end{boxedminipage}
\caption{Sedna exceptions hierarchy}
\label{fig:exceptions}
\end{figure}


\paragraph{SednaException} abstract base exception class. You cannot use it for raising
exceptions.

\paragraph{SednaSystemException} use it when some system error happens. This error means
that the system is completely malfunction. The reaction on this error is hard 
stopping of all components currently running.

\paragraph{SednaSystemEnvException} the same as SednaSystemException except one point.
The output to a user produced as the reaction to this error says that it was 
environment (operating system) fault and the system can not be longer running.
For the previous exception responsibility is on the developers.

\paragraph{SednaUserException} use it when some kind of error happens but this error
is not fatal for the system but rather a recoverable error. Most errors of 
this type are errors caused by users (wrong queries they sends). But they can
be other errors such as ``can not connect to sm because sm is not running"
caused by trn. The reaction on this error is to produce correct message to the
user with an explanation of the problem. Error codes and descriptions of errors
are defined in error.codes file, which is in the same directory as the file you
are reading now.

\paragraph{SednaUserExceptionFnError} use it for errors raised by users (\texttt{fn:error} function).

\paragraph{SednaUserEnvException} the same as \emph{SednaUserException} with a bit different
semantics.  Use this exception to notify the user that the system cannot do
something because of the environment (operating system). You need not define 
error code for this kind of error (it is already defined), but you
rather need to provide a correct description. This exception is needed to
simplify the life of the developers, because it allow them not to define
enormous number of error codes for routine operations such as semaphore
creation.

The example of using this exception is the following:

\smallskip\noindent
\begin{boxedminipage}{\textwidth}
\begin{verbatim}
if (<cannot create semaphore during startup>)
   throw USER_ENV_EXCEPTION(
      "Cannot create semaphore <the name of the semaphore>", false);
\end{verbatim}
\end{boxedminipage}

\paragraph{SednaUserSoftException} this exception class rather differs from other 
exceptions. It is used for correct program termination instead of signaling
some error condition. Its purpose can be clarified by the following example.
Suppose you have to parse command line arguments for already started (and 
initialized) process. If you find some error in command line parameters its your
obligatory to finish the process and bring the error message to the user. If you
just simply write error message to console and exit from the process, it may
crash the system because in this case you avoid process deinitialization, which
is performed by the program code somewhere else in the program. From the other
hand, if you raise SednaUserException then you have to supply error code and,
what is more important, the error message that is identified by this error
code will be enlarged with additional information such as error code and a 
comment. The solution to this problem is to use SednaUserSoftException for raising
the error. In this case process deinitialization will be completed as a part
of the standard error handling mechanism and the user will see exactly the same
error message that you have supplied.

\paragraph{SednaXQueryException} use it within physical plan operations (\texttt{PP}*). It has 
semantic as SednaUserException with better diagnostic. Line of the XQuery query
must be provided with which error is connected. 0 means `I can't say exact line'.

\subsection{Constructing Exception Objects}

For constructing exception objects it is better to use the set of predefined macroses --- consult fig.~\ref{fig:macroses} for details.

\begin{figure}[hbt]
\centering
\begin{boxedminipage}{\textwidth}
\begin{description}
\item[SYSTEM\_EXCEPTION({\it msg})]
\item[SYSTEM\_ENV\_EXCEPTION({\it msg})]
\item[USER\_EXCEPTION({\it code})]
\item[USER\_EXCEPTION2({\it code}, {\it details})]
\item[XQUERY\_EXCEPTION({\it code})]
\item[XQUERY\_EXCEPTION2({\it code}, {\it details})]
\item[USER\_EXCEPTION\_FNERROR({\it ename}, {\it edescr})]
\item[USER\_ENV\_EXCEPTION({\it msg}, {\it rollback})]
\item[USER\_ENV\_EXCEPTION2({\it msg}, {\it expl}, {\it rollback})]
\item[USER\_SOFT\_EXCEPTION({\it msg})]
\end{description}
\end{boxedminipage}
\begin{tabular}{lp{5in}}
\it msg & a textual message (some kind of error description) \\
\it code & the code for user defined error (use constants defined in \verb|error_codes.h|; example is \verb|SE1001|)\\
\it details & details for user error\\
\it ename & error name for fn error\\
\it edescr &error description for fn error\\
\it expl & explanation of error\\
\it rollback & does the error leads to rollback?\\
\end{tabular}
\caption{Constructing exception objects}
\label{fig:macroses}
\end{figure}


\subsection{Error Reporting Format}

Errors could be outputted to the user in the following format -- consult fig.~\ref{fig:output} for details.

\begin{figure}[t]
\centering
\begin{boxedminipage}{\textwidth}
\begin{verbatim}
<sedna-message> ::=
            "SEDNA Message: " <problem> <line-break>
            <description> <line-break>
            [ "Details: " <details> <line-break> ]
            [ "Position: [" <file> ":" <function> ":" <line> "]" <line-break> ]

<problem> ::= "ERROR " <code>
            | "FATAL ERROR"

<description> ::= <string-without-line-break>

<file> ::= <string-without-line-break>

<function> ::= <string-without-line-break>

<line> ::= <integer>

<code> ::= <code-from-(error.codes)>
\end{verbatim}
\end{boxedminipage}
\caption{Error reporting format}
\label{fig:output}
\end{figure}

\end{document}

\end{document}
