
% File:  QuickStart.tex
% Copyright (C) 2004 The Institute for System Programming of the Russian Academy of Sciences (ISP RAS)

\documentclass[a4paper,12pt]{article}
\font\mmmm cmtt10
\makeatletter
\def\verbatim@font{\mmmm}
\makeatother
\usepackage{amsmath}
\usepackage{amssymb}
\usepackage{theorem}
\newtheorem{definition}{Definition}
{\theorembodyfont{\rmfamily} \newtheorem{proof}{Proof}}
\newtheorem{theo}{Theorem}
\newtheorem{note}{Note}

\usepackage[dvips]{epsfig}

\usepackage{multirow}

\title{Sedna Quick Start}
\date{}
%\author{Dmitry Lizorkin}

\begin{document}

\sloppy

\maketitle

Sedna packages come with an example that allows you to investigate the features of Sedna easily. 
Section \ref{sec:how-to-run} describes how to run the example. Section \ref{sec:example-description} gives a brief overview of the example. In case of differences between the Windows and Linux/FreeBSD/MacOS versions of Sedna they are marked out with [win:] and [nix:] respectively.

\section{How to run the example}
\label{sec:how-to-run}
Before runing the example you should install Sedna. See \verb!README! in your Sedna package for the instructions.
In the remainder of this guide \verb!INSTALL_DIR! refers to the directory where Sedna is installed.  
The example is located in the directory:

\begin{verbatim}
[win:] INSTALL_DIR\examples\commandline
[nix:] INSTALL_DIR/examples/commandline
\end{verbatim}

To run the example, type the following commands in the command prompt:
\begin{enumerate}
\item Change the current directory to the directory where the example is located by typing in a command line:
\begin{verbatim}
[win:] cd INSTALL_DIR\examples\commandline
[nix:] cd INSTALL_DIR/examples/commandline
\end{verbatim}
\item Start Sedna by runing the following command: 
\begin{verbatim}
se_gov 
\end{verbatim}
If Sedna is started successfully it prints "\verb!GOVERNOR has been! \verb!started in! \verb!the background mode!".
\item Create a new database \verb!auction! by running the following command: 
\begin{verbatim}
se_cdb auction
\end{verbatim}
If the database is created successfully it prints "\verb!The database 'auction'! \verb!has been! \verb!created successfully!". 
\item Start the \verb!auction! database by running the following command:
\begin{verbatim}
se_sm auction
\end{verbatim}
If the database is started successfully it prints "\verb!SM has been! \verb!started in! \verb!the background mode!".
\item Load the sample XML document into the \verb!auction! database by typing the command:
\begin{verbatim}
se_term -file load_data.xquery auction
\end{verbatim}
If the document is loaded successfully it prints "\verb!UPDATE! \verb!is executed! \verb!successfully!".
\item Now you can execute the sample queries by typing the command:
\begin{verbatim}
se_term -file <query_name>.xquery auction
\end{verbatim}
where \verb!<query_name>.xquery! is the name of a file with the sample query.
For instance, to execute \verb!sample01.xquery! you should type 
\begin{verbatim}
se_term -file sample01.xquery auction
\end{verbatim}
It prints the query result.
\item Stop Sedna by running the following command:
\begin{verbatim}
se_stop
\end{verbatim}
\end{enumerate}


\section{Example Description}
\label{sec:example-description}
The example consists of a sample XML document and a set of sample XQuery queries to this document. The example is located in the directory \verb![win:] INSTALL_DIR\examples\commandline! \verb![nix:] INSTALL_DIR/examples/commandline!. The example is based on the XMark XML benchmark \cite{xmark}.

The sample document named \verb!auction.xml! contains sample information from an Internet auction site. The main elements of the document are: \emph{person}, \emph{bid}, \emph{open auction}, \emph{closed auction}, \emph{item}, \emph{category}, and \emph{mail}. The short description of these elements is as follows. \emph{Items} are the objects that are on for sale or that already have been sold. \emph{Auctions} can be of two types \emph{closed auction} when all items have been sold or \emph{open auction} when there are items on offer. \emph{Persons} make \emph{bids} increasing prices of items and interested in some \emph{categories} of items. Categories are linked into network. There are also some \emph{mails}, concerning items. 

We also provide ten sample XQuery queries to the document, that give a good illustration of the Sedna functionality. The semantics of the queries is given below for your convenience:
\begin{enumerate}
\item Return the name of the person with ID `person0' registered in North
 America.
\item Return the initial increases of all open auctions.
\item Return the number of sold items that cost more than 40.
\item Return the number of items listed on all continents.
\item Return the number of pieces of prose in the database.
\item List the names of items registered in Australia along with their
 descriptions.
\item For each richer-than-average person, list the number of items currently
 on sale whose price does not exceed 0.02 of the person's income.
\item Group customers by their income and output the cardinality of each group.
\item Inserts a new person description into the auction.
\item Delete the person John Smith from auction description.
\end{enumerate}

\begin{thebibliography}{9}
\bibitem{xmark} The XML benchmark project (XMark), www.xml-benchmark.org
\end{thebibliography}

\end{document}

