% QuickStart.tex: Sedna Quick Start
% Copyright (C) 2010 ISP RAS
% The Institute for System Programming of the Russian Academy of Sciences

\documentclass[a4paper,12pt]{article}

\usepackage{alltt}         % Like verbatim but supports commands inside
\usepackage{theorem}
\newtheorem{note}{Note}    % To insert notes
\usepackage{multirow}      % Allows inserting tables
\usepackage{ifpdf}         % Package for conditionals in TeX
\newcommand{\TocAt}[6]{}   % To avoid processing \TocAt by LaTeX

\title{Sedna Quick Start}
\date{}

% Switch for between PDF and other formats to generate bookmarks,
% pretty table of contents and set document's information in PDF
\ifpdf
  \usepackage[colorlinks=true, linkcolor=blue,
              citecolor=blue, urlcolor=blue,
              pdftex,                %%% hyper-references for pdflatex
              bookmarks=true,        %%% generate bookmarks ...
              bookmarksnumbered=true %%% ... with numbers
  ]{hyperref}
  \pdfadjustspacing=1
  \hypersetup{
	pdfauthor   = {Sedna Team},
	pdftitle    = {Sedna Quick Start}
  }
\else
  \usepackage[colorlinks=true, linkcolor=blue,
			  citecolor=blue, urlcolor=blue]{hyperref}
\fi

% Use citemize environment to produce tightly packed lists
\newenvironment{citemize}
{\begin{itemize}
  \setlength{\itemsep}{0pt}
  \setlength{\parskip}{0pt}
  \setlength{\parsep}{0pt}}
{\end{itemize}}


%===============================================================================
%                           Sedna Quick Start
%===============================================================================
\begin{document}
\sloppy
\maketitle

Sedna packages come with an example that allows you to investigate the features
of Sedna easily. Section \ref{sec:how-to-run} describes how to run the example.
Section \ref{sec:example-description} gives a brief overview of the example. In
case of differences between the Windows and Linux/FreeBSD/MacOS/Solaris versions
of Sedna they are marked with [win:] and [nix:] respectively.

\section{How To Run The Example}
\label{sec:how-to-run}

Before runing the example you should install Sedna. See \verb!README! in your
Sedna package for the instructions. In the remainder of this guide
\verb!INSTALL_DIR! refers to the directory where Sedna is installed. The example
is located in the directory:

\begin{verbatim}
[win:] INSTALL_DIR\examples\commandline
[nix:] INSTALL_DIR/examples/commandline
\end{verbatim}

To run the example, type the following commands in the command prompt:
\begin{enumerate}
\item Change the current directory to the directory where the example is located
by typing in a command line:
\begin{verbatim}
[win:] cd INSTALL_DIR\examples\commandline
[nix:] cd INSTALL_DIR/examples/commandline
\end{verbatim}

\item Start Sedna by runing the following command:
\begin{verbatim}
se_gov
\end{verbatim}
If Sedna is started successfully it prints "\verb!GOVERNOR has been!
\verb!started in! \verb!the background mode!".

\textbf{Important note:} since version 3.5 Sedna server by default listens
on 'localhost' and allows only local clients. If you want to work with Sedna
remotely make sure that \verb!-listen-address! parameter value is
adjusted properly. See Sedna Administration Guide for the details on how to set
this option.

\item Create a new database \verb!auction! by running the following command:
\begin{verbatim}
se_cdb auction
\end{verbatim}
If the database is created successfully it prints "\verb!The database 'auction'!
\verb!has been! \verb!created successfully!".

\item Start the \verb!auction! database by running the following command:
\begin{verbatim}
se_sm auction
\end{verbatim}
If the database is started successfully it prints "\verb!SM has been!
\verb!started in! \verb!the background mode!".

\item Load the sample XML document into the \verb!auction! database by typing
the command:
\begin{verbatim}
se_term -file load_data.xquery auction
\end{verbatim}
If the document is loaded successfully it prints "\verb!Bulk! \verb!load!
\verb!succeeded!".

\item Now you can execute the sample queries by typing the command:
\begin{verbatim}
se_term -file <query_name>.xquery auction
\end{verbatim}
where \verb!<query_name>.xquery! is the name of a file with the sample query.
For instance, to execute \verb!sample01.xquery! you should type:
\begin{verbatim}
se_term -file sample01.xquery auction
\end{verbatim}
It prints the query result.

\item Stop Sedna by running the following command:
\begin{verbatim}
se_stop
\end{verbatim}
\end{enumerate}


%===============================================================================
%                   Sedna Quick Start: Example Description
%===============================================================================
\section{Example Description}
\label{sec:example-description}

The example consists of a sample XML document and a set of sample XQuery queries
to this document. The example is located in the directory:
\begin{verbatim}
[win:] INSTALL_DIR\examples\commandline
[nix:] INSTALL_DIR/examples/commandline
\end{verbatim}
The example is based on the XMark XML benchmark \cite{xmark}.

The sample document named \verb!auction.xml! contains sample information from an
Internet auction site. The main elements of the document are: \emph{person},
\emph{bid}, \emph{open auction}, \emph{closed auction}, \emph{item},
\emph{category}, and \emph{mail}. The short description of these elements is as
follows. \emph{Items} are the objects that are on for sale or that already have
been sold. \emph{Auctions} can be of two types \emph{closed auction} when all
items have been sold or \emph{open auction} when there are items on offer.
\emph{Persons} make \emph{bids} increasing prices of items and interested in
some \emph{categories} of items. Categories are linked into network. There are
also some \emph{mails}, concerning items.

We also provide ten sample XQuery queries to the document, that give a good
illustration of the Sedna functionality. The semantics of the queries is given
below for your convenience:

\begin{enumerate}
\item Return the name of the person with ID `person0' registered in North
 America.
\item Return the initial increases of all open auctions.
\item Return the number of sold items that cost more than 40.
\item Return the number of items listed on all continents.
\item Return the number of pieces of prose in the database.
\item List the names of items registered in Australia along with their
 descriptions.
\item For each richer-than-average person, list the number of items currently
 on sale whose price does not exceed 0.02 of the person's income.
\item Group customers by their income and output the cardinality of each group.
\item Inserts a new person description into the auction.
\item Delete the person John Smith from auction description.
\end{enumerate}

%===============================================================================
%                   Sedna Quick Start: Troubleshooting
%===============================================================================
\section{Troubleshooting}
\label{sec:troubleshooting}

If you face any problems in using our product we would be glad to answer your
questions and we would be pleased to get your bug-reports. But we will very
appreciate if you read FAQ\cite{link:FAQ} and Sedna
documentation\cite{link:Documentation} firstly. You can also find the
documentation on your Sedna version in your Sedna install directory (or your
build directory if you have compiled it yourself).

You can contact us via mailing list\cite{link:mailing-list} with your questions.
If you think that you have found a bug you can also use our bug
tracker\cite{link:bug-tracker} but if you are not sure you still may contact us
via mailing list. In both cases we expect you to tell us the following:

\begin{enumerate}
\item What version of Sedna do you use.
\item What operating system is used to run Sedna.
\item Most likely we will need to see your event.log (it's located in your data
directory) to answer your question so it would be better if you attach it to
your question.
\end{enumerate}



\begin{thebibliography}{9}
\bibitem{xmark} The XML benchmark project (XMark), \url{www.xml-benchmark.org}

\bibitem{link:FAQ}
Frequently Asked Questions,
\url{http://sedna.org/faq.html}

\bibitem{link:Documentation}
Sedna Native XML Database System Documetation,
\url{http://sedna.org/documentation.html}

\bibitem{link:mailing-list}
Sedna DBMS mailing list,
\url{http://lists.sourceforge.net/lists/listinfo/sedna-discussion}

\bibitem{link:bug-tracker}
Sedna DBMS bug tracker,
\url{http://sourceforge.net/tracker2/?group_id=129076&atid=713730}

\end{thebibliography}

\end{document}

