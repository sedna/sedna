% ClientServerProtocol.tex: Sedna Client/Server Protocol
% Copyright (C) 2010 ISP RAS
% The Institute for System Programming of the Russian Academy of Sciences

\documentclass[a4paper,12pt]{article}

\usepackage{alltt}         % Like verbatim but supports commands inside
\usepackage{theorem}
\newtheorem{note}{Note}    % To insert notes
\usepackage{multirow}      % Allows inserting tables
\usepackage{ifpdf}         % Package for conditionals in TeX
\newcommand{\TocAt}[6]{}   % To avoid processing \TocAt by LaTeX

\title{Sedna Native XML Database Client/Server Protocol}
\date{}

% Switch for between PDF and other formats to generate bookmarks,
% pretty table of contents and set document's information in PDF
\ifpdf
  \usepackage[colorlinks=true, linkcolor=blue, 
              citecolor=blue, urlcolor=blue,
              pdftex,                %%% hyper-references for pdflatex
              bookmarks=true,        %%% generate bookmarks ...
              bookmarksnumbered=true %%% ... with numbers
  ]{hyperref}
  \pdfadjustspacing=1
  \hypersetup{
	pdfauthor   = {Sedna Team},
	pdftitle    = {Sedna Native XML Database Client/Server Protocol}
  }
\else
  \usepackage[colorlinks=true, linkcolor=blue,
			  citecolor=blue, urlcolor=blue]{hyperref}
\fi

% Use citemize environment to produce tightly packed lists
\newenvironment{citemize}
{\begin{itemize}
  \setlength{\itemsep}{0pt}
  \setlength{\parskip}{0pt}
  \setlength{\parsep}{0pt}}
{\end{itemize}}


%===============================================================================
%                Sedna Client/Server Protocol: Introduction
%===============================================================================
\begin{document}
\sloppy
\maketitle

\TocAt*{section,subsection,subsubsection}
\TocAt*{subsection,subsubsection}

\tableofcontents
\newpage

\section{Introduction}
This document describes details of the message-based protocol Sedna XML Database
server uses for communication with clients through the TCP/IP sockets. Higher
level application programming interfaces are built over this protocol.

This document describes versions 1.0, 2.0, 3.0 and 4.0 of the protocol. It
consists of three parts: section \ref{sec:MessageStructure} describes the basic
message structure, section \ref{sec:MessageFlow} defines communication protocol
and \ref{sec:MessageFormats}-th section describes the detailed format of each
message.


%===============================================================================
%              Sedna Client/Server Protocol: Known Implementations 
%===============================================================================
\subsection{Known Implementations}

The known implementations of the protocol include:
\begin{itemize}
\item \href{http://www.cfoster.net/sedna/xqj/}{XQJ Driver} by Charles Foster
(version 4.0);
\item \href{http://www.cfoster.net/sedna/}{XML:DB API Driver} by Charles Foster
(version 4.0);
\item Java Driver included in distribution (version 2.0);
\item Scheme Driver included in distribution (version 2.0);
\item Terminal (\verb!se_trm!) included in distribution (version 3.0).
\end{itemize}

\newpage


%===============================================================================
%              Sedna Client/Server Protocol: Message Structure 
%===============================================================================
\section{Message Structure}
\label{sec:MessageStructure}

In messages values of the following three data types are used:
\begin{enumerate}
\item\verb!byte! -- one byte;
\item\verb!int! -- four bytes presented in the network byte order (most
significant byte first);
\item\verb!string! -- has the following structure: the first byte identifies the
string format, the next four bytes (int) specify the length of the string in
bytes and the next 'length' bytes is the string. The only supported string
format is C-string without trailing null character. The first byte equal zero
identifies C-string.
\end{enumerate}

The common message structure is as follows:
\begin{citemize}
\item the first four bytes (int) is instruction;
\item the next four bytes (int) is the length of a body in bytes;
\item the next 'length' bytes is the body.
\end{citemize}
The body of the message is determined by the instruction. In general the body of
all messages is a sequence of values of the three types listed above. The
position of a value in the sequence determines its meaning.

In the current version of Sedna the size of the message body must not exceed
10240 bytes.

\newpage


%===============================================================================
%                        Message Flow: Start-Up 
%===============================================================================
\section{Message Flow}
\label{sec:MessageFlow}

\subsection{Start-Up}

To begin a session, a client creates a connection to the server and sends a
startup message. The server launches a new process that is associated with the
session. If launching a new process causes some errors, the server sends the
\verb!se_ErrorResponse! message, if not, it sends the
\verb!se_SendSessionParameters! message to the client. Then client sends the
session parameters in the \verb!se_SessionParameters! message. This message
includes the particular protocol version to be used, the name of the user and of
the database the user wants to connect to. The server then uses this information
to determine whether the connection is acceptable. If not, it sends the error
message (\verb!se_ErrorResponse!). If the connection is acceptable the server
then sends an authentication request message, to which the client must reply
with an appropriate authentication response message. In principle the
authentication request/response cycle could require multiple iterations, but the
present authentication method uses exactly one request
(\verb!se_SendAuthParameters!) and response
(\verb!se_AuthenticationParameters!). The authentication cycle ends with the
server either rejecting the connection attempt (\verb!se_AuthenticationFailed!),
or sending \verb!se_AuthenticationOk!.

The possible instructions from the client in this phase are:
\begin{citemize}
\item \verb!se_Start-Up!. Does not contain the body.
\item \verb!se_SessionParameters!. The body contains the protocol version, user
name and db name.
\item \verb!se_AuthenticationParameters!. The body contains password.
\end{citemize}

The possible instructions from the server in this phase are:
\begin{citemize}
\item \verb!se_SendSessionParameters!. Does not contain the body.
\item \verb!se_SendAuthParameters!. Does not contain the body.
\item \verb!se_AuthenticationOK!. Does not contain the body.
\item \verb!se_AuthenticationFailed!. Body contains info.
\item \verb!se_ErrorResponse!. Body contains info.
\end{citemize}


%===============================================================================
%                        Message Flow: Transactions 
%===============================================================================
\subsection{Transactions}

After the start-up phase has succeeded and session is begun, client can run zero
or more transactions in the session.

Transactions must be run sequentially, that is the client must commit a
transaction before starting a new one.

To begin a transaction client sends the \verb!se_BeginTransaction! message. If
the transaction begins Ok, the server answers \verb!se_BeginTransactionOk!. If
the transaction fails to begin, the server answers
\verb!se_BeginTransactionFailed!.

To commit the transaction client sends the \verb!se_CommitTransaction! message.
If the transaction commits Ok, the server answers \verb!se_CommitTransactionOk!.
If the transaction fails to commit, the server does rollback for the transaction
and answers \verb!se_CommitTransactionFailed!.

To rollback the transaction client sends the \verb!se_RollbackTransaction!
message. If the transaction rollbacks Ok, the server answers
\verb!se_RollbackTransactionOk!. If the transaction failed to rollback, the
server sends \verb!se_RollbackTransactionFailed! and closes the session.

The possible instructions from the client in this phase are:
\begin{citemize}
\item \verb!se_BeginTransaction!. Does not contain a body.
\item \verb!se_CommitTransaction!. Does not contain a body.
\item \verb!se_RollbackTransaction!. Does not contain a body.
\end{citemize}

The possible instructions from the server in this phase are:
\begin{citemize}
\item \verb!se_BeginTransactionOk!. Does not contain a body.
\item \verb!se_BeginTransactionFailed!. The body contains the error code and
error info.
\item \verb!se_CommitTransactionOk!. Does not contain a body.
\item \verb!se_CommitTransactionFailed!. The body contains the error code and
error info.
\item \verb!se_RollbackTransactionOk!. Does not contain a body.
\item \verb!se_RollbackTransactionFailed!. The body contains the error code and
error info.
\item \verb!se_ErrorResponse!. Body contains info.
\end{citemize}


%===============================================================================
%                         Message Flow: Session Options 
%===============================================================================
\subsection{Session Options}

Since version 3.0 of the Sedna Client-Server protocol it is possible to set
session options.

There are a number of session options. Session options can be set at any moment
during the session except the period when session is in a query evaluation phase
(executing or passing result data to a client).

To set one or more options client must send \verb!se_SetSessionOptions! message.
If options were set successfully server sends \verb!se_SetSessionOptionsOk!
message to the client. Otherwise server sends \verb!se_ErrorResponse! to the
client.

To reset options to their default values client must send
\verb!se_ResetSessionOptions!. If options were reset successfully server sends
\verb!se_ResetSessionOptionsOk! message to the client. Otherwise server sends
\verb!se_ErrorResponse! to the client.

The possible instructions from client are:
\begin{citemize}
\item \verb!se_SetSessionOptions!. The body contains any number of pairs: option
id followed by option value. Option id is int, option value is string.
\item \verb!se_ResetSessionOptions!. Does not contain a body.
\end{citemize}

The possible instructions from the server are:
\begin{citemize}
\item \verb!se_SetSessionOptionsOk!. Does not contain a body.
\item \verb!se_ResetSessionOptionsOk!. Does not contain a body.
\item \verb!se_ErrorResponse!. Body contains info.
\end{citemize}

Possible option ids:
\begin{citemize}
\item \verb!SEDNA_DEBUG_OFF! -- turns off query debug mode. Query debug mode is
off be default. (See "Debug features" section of the Sedna Programmer's Guide
for details.) This option does not contain a value (there must be a string of
zero length in a message body);
\item \verb!SEDNA_DEBUG_ON! -- turns on query debug mode. This option does not
contain a value (there must be a string of zero length in a message body).
\end{citemize}


%===============================================================================
%                        Message Flow: Query Execution 
%===============================================================================
\subsection{Query Execution}

Queries are executed via different subprotocols depending on the type of the
query and the query length. There are three types of queries: query, update,
bulk load.

If the query is not longer than the message body (10240 bytes) the client sends
the Execute message that contains a query statement to the server. If the query
is longer than the message body the client must send the query in parts: each
part is in the body of \verb!se_ExecuteLong! message. After all parts of the
query are sent, client must send \verb!se_LongQueryEnd! message (with the empty
body), thus to make the server understand that the whole query is sent.

The server analyzes the query text to identify the type of the query and runs
the corresponding subprotocol. The following sections describe these
subprotocols.

\subsubsection{Querying Data }

The client sends a query in the \verb!se_Execute! message, or in the
\verb!se_ExecuteLong! messages plus the \verb!se_LongQueryEnd! message. The
first byte of the query text is a format of the result to obtain - SXML
\cite{paper:sxml} or XML. Use \verb!0! if you want to get the result of the
query in XML, \verb!1! - to get the result in SXML.

\begin{note}
Since Protocol Version 4.0 it is client's task to indent and pretty print
output.
\end{note}

The server processes the query. If the query succeeds the server sends the
\verb!se_QuerySucceeded! message to the client and then sends a number of
messages that contain the first item of the result sequence and query debug
information (if any) to the client in the way described below.

When the client needs to get the next item of the result sequence it sends the
\verb!se_GetNextItem! message. The server then sends the next item of the result
sequence.

The way server sends items depends on the protocol version:

\textbf{Protocol Versions 1.0, 2.0, 3.0:} Server may send every item in parts.
Every part of the item is enveloped in the \verb!se_ItemPart! message. When the
whole item is sent, the server sends the \verb!se_ItemEnd! message or
\verb!se_ResultEnd! message if it was the last item.

\textbf{Protocol Version 4.0:} Server sends items in the following way.
\verb!se_ItemStart! message is sent in the first place. It contains type and
content of the item being sent. If content is too long to be sent within one
message, server may send it in parts. Every part of the item is enveloped in the
\verb!se_ItemPart! message. When the whole item is sent, the server sends the
\verb!se_ItemEnd! message or \verb!se_ResultEnd! message if it was the last
item.

When the result sequence has ended, server on receiving \verb!se_GetNextItem!
from client sends the ResultEnd message without prior sending the
\verb!se_ItemPart! message.

While sending result data server may also send any number of messages
\verb!se_DebugInfo! containing debug information if there is any.

\begin{note}
Client debug information is supported in Sedna Client/Server Protocol since
version 2.0.
\end{note}

Client is not required to get all the items of the result sequence. It can send
next query for execution before all the items of the result sequence are
received from the server.

If the query failed, the server sends message \verb!se_QueryFailed! to the client.

The possible instructions from the client in this phase are:
\begin{citemize}
\item \verb!se_Execute!. The body contains an XQuery query text.
\item \verb!se_ExecuteLong!. The body contains a part of a long XQuery query.
\item \verb!se_LongQueryEnd!. Does not contain a body.
\item \verb!se_GetNextItem!. Does not contain a body.
\end{citemize}

The possible instructions from the server in this phase are:
\begin{citemize}
\item \verb!se_QuerySucceeded!. Does not contain a body.
\item \verb!se_QueryFailed!. The body contains the error code and error info.
\item \verb!se_DebugInfo!. The body contains the debug type and debug info.
(Since version 2.0)
\item \verb!se_ItemStart!. The body contains type of the item being sent and the
part of the item. (Since version 4.0)
\item \verb!se_ItemPart!. The body contains the part of the item.
\item \verb!se_ItemEnd!. Does not contain a body.
\item \verb!se_ErrorResponse!. Body contains info.
\end{citemize}

\subsubsection{Updating Data}

The client sends the \verb!se_Execute! message (or the \verb!se_ExecuteLong!
messages plus the \verb!se_LongQueryEnd! message) that contains an update
statement. The server processes the update. If the update succeeded the server
sends the \verb!se_UpdateSucceeded! message to the client. If the update failed,
the server sends the \verb!se_UpdateFailed! message.

Before sending \verb!se_UpdateSucceeded! or \verb!se_UpdateFailed! message sever
may send any number of \verb!se_DebugInfo! messages if there is any debug
information.

The possible instructions from the client in this phase are:
\begin{citemize}
\item \verb!se_Execute!. The body contains an update statement.
\item \verb!se_ExecuteLong!. The body contains a part of a long XQuery query.
\item \verb!se_LongQueryEnd!. Does not contain a body.
\end{citemize}

The possible instructions from the server in this phase are:
\begin{citemize}
\item \verb!se_UpdateSucceeded!. Does not contain a body.
\item \verb!se_UpdateFailed!. The body contains the error code and error info.
\item \verb!se_DebugInfo!. The body contains the debug type and debug info.
\item \verb!se_ErrorResponse!. Body contains info.
\end{citemize}

\subsubsection{Bulk Load}

The client sends the \verb!se_Execute! message that contains a bulk load
statement. The server picks out the name of the file and sends the
\verb!se_BulkLoadFileName! message that contains the name.

Since version 2.0 of the Sedna Client-Server protocol server can send multiple
\verb!se_BulkLoadFileName! messages if there were multiple file names in a
query. This currently can happen in \verb!LOAD MODULE! statement.

The client reads the file specified. If there is no such file or some access
errors occur, the client sends \verb!se_BulkLoadError! message to the server.
Else the client transfers the data from the file to the server by portions. Each
portion is sent in the \verb!se_BulkLoadPortion! message.

When the whole file is sent, the client sends the \verb!se_BulkLoadEnd! message.
The server answers with the \verb!se_BulkLoadSucceeded! or
\verb!se_BulkLoadFailed! message.

The possible instructions from the client in this phase are:
\begin{citemize}
\item \verb!se_Execute!. The body contains a query for bulk load.
\item \verb!se_BulkLoadError!. Does not contain the body.
\item \verb!se_BulkLoadPortion!. The body contains portion of data.
\item \verb!se_BulkLoadEnd!. Does not contain the body.
\end{citemize}

The possible instructions from the server in this phase are:
\begin{citemize}
\item \verb!se_BulkLoadFileName!. The body contains file name.
\item \verb!se_BulkLoadSucceeded!. Does not contain the body.
\item \verb!se_BulkLoadFailed!. The body contains the error code and error info.
\item \verb!se_ErrorResponse!. Body contains info.
\end{citemize}


%===============================================================================
%                         Message Flow: Termination 
%===============================================================================
\subsection{Termination}

Termination can be initiated by the client (for example when it closed the
session) or by the server (for example in case of an administrator-commanded
database shutdown or some failure).

The normal termination procedure is that the client closes the session after
transaction commit. In this case the client sends the \verb!se_CloseConnection!
message. The server processes its closing procedure. If no errors occur the
server sends the \verb!se_CloseConnectionOk! message and closes the connection.

If the client sends the \verb!se_CloseConnection! message before committing the
on-going transaction, the server does rollback for the transaction, sends the
\verb!se_TransactionRollbackBeforeClose! message and closes the connection.

If on receipt of the \verb!se_CloseConnection! message, some errors on server
occur the server sends the \verb!se_ErrorResponse! message and closes.

While an administrator-commanded database shutdown or some failure occurs the
server may disconnect without any client request to do so. Before closing the
connection the server sends the \verb!se_ErrorResponse! message that contains
error code and error info.

The possible instructions from the client in this phase are:
\begin{citemize}
\item \verb!se_CloseConnection!. Does not contain the body.
\end{citemize}

The possible instructions from the server in this phase are:
\begin{citemize}
\item \verb!se_CloseConnectionOk!. Does not contain the body.
\item \verb!se_TransactionRollbackBeforeClose!. Does not contain the body.
\item \verb!se_ErrorResponse!. Body contains info.
\end{citemize}


%===============================================================================
%                     Message Flow: Server Error Handling 
%===============================================================================
\subsection{Server Error Handling}

In all phases of client/server interaction an error can occur on the server. In
this case the server answers to a client request message by sending the
\verb!se_ErrorResponse! message that contains the error code and the error info.

\begin{citemize}
\item \verb!se_ErrorResponse!. The body contains the error code and the error
info.
\end{citemize}

\newpage

%===============================================================================
%                             Message Formats 
%===============================================================================
\section{Message Formats}
\label{sec:MessageFormats}

This section describes the detailed format of each message. Each is marked to
indicate that it may be sent either by a client (C), or a server (S).

\begin{verbatim}
se_Start-Up (C).
  head:
   110 (int)
   body length = 0(int)
  body:
   empty
\end{verbatim}

\begin{verbatim}
se_SessionParameters (C).
   head:
	120 (int)
	body length (int)
  body:
   major protocol version number (byte);
   minor protocol version number (byte);
   user name (string);
   database name (string);
\end{verbatim}

\begin{verbatim}
se_AuthenticationParameters (C).
  head:
   130 (int)
   body length (int)
  body:
   password (string)
\end{verbatim}

\begin{verbatim}
se_SendSessionParameters (S).
  head:
   140 (int)
   body length = 0 (int)
  body:
   empty
\end{verbatim}

\begin{verbatim}
se_SendAuthParameters (S).
  head:
   150 (int)
   body length = 0 (int)
  body:
   empty
\end{verbatim}

\begin{verbatim}
se_AuthenticationOK (S).
  head:
   160 (int)
   body length = 0 (int)
  body:
   empty
\end{verbatim}

\begin{verbatim}
se_AuthenticationFailed (S).
  head:
   170 (int)
   body length (int)
  body:
   error code (int)
   error info (string)
\end{verbatim}

\begin{verbatim}
se_ErrorResponse (S).
  head:
   100 (int)
   body length (int)
  body:
   error code (int)
   error info (string)
\end{verbatim}

\begin{verbatim}
se_BeginTransaction (C).
  head:
   210 (int)
   body length = 0(int)
  body:
   empty
\end{verbatim}

\begin{verbatim}
se_CommitTransaction (C).
  head:
   220 (int)
   body length = 0(int)
  body:
   empty
\end{verbatim}

\begin{verbatim}
se_RollbackTransaction (C).
  head:
   225 (int)
   body length = 0(int)
  body:
   empty
\end{verbatim}

\begin{verbatim}
se_BeginTransactionOk (S).
  head:
   230 (int)
   body length = 0(int)
  body:
   empty
\end{verbatim}

\begin{verbatim}
se_BeginTransactionFailed (S).
  head:
   240 (int)
   body length (int)
  body:
   error code (int)
   error info (string)
\end{verbatim}

\begin{verbatim}
se_CommitTransactionOk (S).
  head:
   250 (int)
   body length = 0(int)
  body:
   empty
\end{verbatim}

\begin{verbatim}
se_CommitTransactionFailed (S).
  head:
   260 (int)
   body length (int)
  body:
   error code (int)
   error info (string)
\end{verbatim}

\begin{verbatim}
se_RollbackTransactionOk (S).
  head:
   255 (int)
   body length = 0(int)
  body:
   empty
\end{verbatim}

\begin{verbatim}
se_RollbackTransactionFailed (S).
  head:
   265 (int)
   body length (int)
  body:
   error code (int)
   error info (string)
\end{verbatim}

\begin{verbatim}
se_Execute (C).
  head:
   300 (int)
   body length (int)
  body:
   result format (byte) + query text (string)
\end{verbatim}

\begin{verbatim}
se_ExecuteLong (C).
  head:
   301 (int)
   body length (int)
  body:
   result format (byte) + query text (string)
\end{verbatim}

\begin{verbatim}
se_LongQueryEnd (C).
  head:
   302 (int)
   body length = 0 (int)
  body:
   empty
\end{verbatim}

\begin{verbatim}
se_GetNextItem (C).
  head:
   310 (int)
   body length = 0 (int)
  body:
   empty
\end{verbatim}

\begin{verbatim}
se_QuerySucceeded (S).
  head:
   320 (int)
   body length = 0(int)
  body:
   empty
\end{verbatim}

\begin{verbatim}
se_DebugInfo (S).
  head:
   325 (int)
   body length (int)
  body:
   debug type (int)
   debug info (string)
\end{verbatim}

\begin{verbatim}
se_QueryFailed (S).
  head:
   330 (int)
   body length (int)
  body:
   error code (int)
   error info (string)
\end{verbatim}

\begin{verbatim}
se_UpdateSucceeded (S).
  head:
   340 (int)
   body length = 0(int)
  body:
   empty
\end{verbatim}

\begin{verbatim}
se_UpdateFailed (S).
  head:
   350 (int)
   body length (int)
  body:
   error code (int)
   error info (string)
\end{verbatim}

\begin{verbatim}
se_ItemStart (S).
  head:
   355 (int)
   body length (int)
  body:
   item class (byte) - see below possible values of this field
   item type (byte)  - see below possible values of this field
   URL flag (byte)   - either 0 or 1, determines if URL field is empty
   [URL (string)]    - optional, URL of the item
   result part (string)
\end{verbatim}

\begin{verbatim}
Item class enumeration (see se_ItemStart message )is defined as follows:
enum se_item_class  {
    se_atomic    = 1,   //item type defines atomic type
    se_document  = 2,
    se_element   = 3,   //item type defines atomic type
    se_attribute = 4,   //item type defines atomic type
    se_namespace = 5,
    se_pi        = 6,
    se_comment   = 7,
    se_text      = 8
};
\end{verbatim}

\begin{verbatim}
Item type enumeration (see se_ItemStart message )is defined as follows:
enum se_item_type {
    /* Abstract base types */
    se_anyType            = 0,
    se_anySimpleType      = 1,
    se_anyAtomicType      = 2,
    /* Built-in simple, non-atomic types */
    se_IDREFS             = 3,
    se_NMTOKENS           = 4,
    se_ENTITIES           = 5,
    /* Built-in complex types */
    se_untyped            = 6,
    /* Built-in atomic types (Primitive types) */
    se_dateTime           = 10,
    se_date               = 11,
    se_time               = 12,
    se_duration           = 13,
    se_yearMonthDuration  = 14,
    se_dayTimeDuration    = 15,
    se_gYearMonth         = 16,
    se_gYear              = 17,
    se_gMonthDay          = 18,
    se_gDay               = 19,
    se_gMonth             = 20,
    se_float              = 21,
    se_double             = 22,
    se_decimal            = 23,
    se_integer            = 24,
    se_boolean            = 25,
    se_untypedAtomic      = 26,
    se_string             = 27,
    se_base64Binary       = 28,
    se_hexBinary          = 29,
    se_anyURI             = 30,
    se_QName              = 31,
    se_NOTATION           = 32,
    /* Types derived from xs:string */
    se_normalizedString   = 41,
    se_token              = 42,
    se_language           = 43,
    se_NMTOKEN            = 44,
    se_Name               = 45,
    se_NCName             = 46,
    se_ID                 = 47,
    se_IDREF              = 48,
    se_ENTITY             = 49,
    /* Types derived from xs:integer */
    se_nonPositiveInteger = 50,
    se_negativeInteger    = 51,
    se_long               = 52,
    se_int                = 53,
    se_short              = 54,
    se_byte               = 55,
    se_nonNegativeInteger = 56,
    se_unsignedLong       = 57,
    se_unsignedInt        = 58,
    se_unsignedShort      = 59,
    se_unsignedByte       = 60,
    se_positiveInteger    = 61
};
\end{verbatim}

\begin{verbatim}
se_ItemPart (S).
  head:
   360 (int)
   body length (int)
  body:
   result part (string)
\end{verbatim}

\begin{verbatim}
se_ItemEnd (S).
  head:
   370 (int)
   body length = 0(int)
  body:
   empty
\end{verbatim}

\begin{verbatim}
se_ResultEnd (S).
  head:
   375 (int)
   body length = 0(int)
  body:
   empty
\end{verbatim}

\begin{verbatim}
se_BulkLoadError (C).
  head:
   400 (int)
   body length (int)
  body:
   error code (int)
   error info (string)
\end{verbatim}

\begin{verbatim}
se_BulkLoadPortion (C).
  head:
   410 (int)
   body length (int)
  body:
   data portion (string)
\end{verbatim}

\begin{verbatim}
se_BulkLoadEnd (C).
  head:
   420 (int)
   body length = 0 (int)
  body:
   empty
\end{verbatim}

\begin{verbatim}
se_BulkLoadFileName (S).
  head:
   430 (int)
   body length (int)
  body:
   file name (string)
\end{verbatim}

\begin{verbatim}
se_BulkLoadFromStream (S).
  head:
   431 (int)
   body length = 0(int)
  body:
   empty
\end{verbatim}

\begin{verbatim}
se_BulkLoadSucceeded (S).
  head:
   440 (int)
   body length = 0(int)
  body:
   empty
\end{verbatim}

\begin{verbatim}
se_BulkLoadFailed (S).
  head:
   450 (int)
   body length(int)
  body:
   error code (int)
   error info (string)
\end{verbatim}

\begin{verbatim}
se_ShowTime (C).
  head:
   451 (int)
   body length = 0(int)
  body:
   empty
\end{verbatim}

\begin{verbatim}
se_LastQueryTime (S).
  head:
   452 (int)
   body length(int)
  body:
   time (string)
\end{verbatim}

\begin{verbatim}
se_CloseConnection (C).
  head:
   500 (int)
   body length = 0(int)
  body:
   empty
\end{verbatim}

\begin{verbatim}
se_CloseConnectionOk (S).
  head:
   510 (int)
   body length = 0(int)
  body:
   empty
\end{verbatim}

\begin{verbatim}
se_TransactionRollbackBeforeClose (S).
  head:
   520 (int)
   body length = 0(int)
  body:
   empty
\end{verbatim}

\begin{verbatim}
se_SetSessionOptions (C).
  head:
   530 (int)
   body length (int)
  body:
   any number of pairs: option id (int), option value (string)
\end{verbatim}

\begin{verbatim}
se_SetSessionOptionsOk (S).
  head:
   540 (int)
   body length = 0(int)
  body:
   empty
\end{verbatim}

\begin{verbatim}
se_ResetSessionOptions (C).
  head:
   550 (int)
   body length = 0(int)
  body:
   empty
\end{verbatim}

\begin{verbatim}
se_ResetSessionOptionsOk (S).
  head:
   560 (int)
   body length = 0(int)
  body:
   empty
\end{verbatim}

\begin{thebibliography}{1}

\bibitem{paper:sxml}
Oleg Kiselyov.
``SXML Specification, Revision 3.0'',
\url{http://www.okmij.org/ftp/Scheme/SXML.html}
\end{thebibliography}

\end{document}

